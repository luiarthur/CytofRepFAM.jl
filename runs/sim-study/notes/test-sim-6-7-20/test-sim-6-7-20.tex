%{{{1
\documentclass[10pt]{article} % 12-point font

\usepackage[margin=1cm]{geometry} % set page to 1-inch margins
\usepackage{amsmath} % for math
\usepackage{amssymb} % like \Rightarrow
\setlength\parindent{0pt} % Suppresses the indentation of new paragraphs.
\pagenumbering{gobble} % suppress page numbers

% Big display
\newcommand{\ds}{ \displaystyle }
% Parenthesis
\newcommand{\norm}[1]{\left\lVert#1\right\rVert}
\newcommand{\p}[1]{\left(#1\right)}
\newcommand{\bk}[1]{\left[#1\right]}
\newcommand{\bc}[1]{ \left\{#1\right\} }
\newcommand{\abs}[1]{ \left|#1\right| }
% Derivatives
\newcommand{\df}[2]{ \frac{d#1}{d#2} }
\newcommand{\ddf}[2]{ \frac{d^2#1}{d{#2}^2} }
\newcommand{\pd}[2]{ \frac{\partial#1}{\partial#2} }
\newcommand{\pdd}[2]{\frac{\partial^2#1}{\partial{#2}^2} }
% Distributions
\newcommand{\Normal}{ \text{Normal} }
\newcommand{\Beta}{ \text{Beta} }
\newcommand{\Gam}{ \text{Gamma} }
\newcommand{\InvGamma}{ \text{Inv-Gamma} }
\newcommand{\Uniform}{ \text{Uniform} }
\def\Dir{\text{Dirichlet}}
\def\TN{\text{TN}}
% Statistics
\newcommand{\E}{\text{E}}
\newcommand{\iid}{\overset{iid}{\sim}}
\newcommand{\ind}{\overset{ind}{\sim}}

% Graphics
\usepackage{graphicx}  % for figures
\usepackage{float} % Put figure exactly where I want [H]

% Uncomment if using bibliography
% Bibliography
\usepackage{natbib}
\bibliographystyle{plainnat}

% Adds settings for hyperlinks. (Mainly for table of contents.)
\usepackage{hyperref}
\hypersetup{
  pdfborder={0 0 0} % removes red box from links
}
%}}}1


% Macros for this project
\def\true{\text{TRUE}}
\usepackage[dvipsnames,usenames]{color}
\newcommand{\bbh}{\color{blue}\textbf}  % blue bold highlight
\newcommand{\bch}{\color{blue}\it}  % blue italics
\newcommand{\ech}{\color{black}\rm}
\def\imgdir{../../results/test-sim-6-7-20}
\newcommand{\imgtemplate}[3]{
  \begin{tabular}{cccccc}
    {\tiny (a) Data1, MCMC1, $\phi=0$} &
    {\tiny (b) Data1, MCMC1, $\phi=1$}&
    {\tiny (c) Data1, MCMC1, $\phi=10$} &
    {\tiny (d) Data1, MCMC2, $\phi=0$} &
    {\tiny (e) Data1, MCMC2, $\phi=1$} &
    {\tiny (f) Data1, MCMC2, $\phi=10$} \\
    %
    \includegraphics[scale=#2]{\imgdir/dataseed1-mcmcseed1-phi0.0#3/img/#1} &
    \includegraphics[scale=#2]{\imgdir/dataseed1-mcmcseed1-phi1.0#3/img/#1} &
    \includegraphics[scale=#2]{\imgdir/dataseed1-mcmcseed1-phi10.0#3/img/#1} &
    %
    \includegraphics[scale=#2]{\imgdir/dataseed1-mcmcseed2-phi0.0#3/img/#1} &
    \includegraphics[scale=#2]{\imgdir/dataseed1-mcmcseed2-phi1.0#3/img/#1} &
    \includegraphics[scale=#2]{\imgdir/dataseed1-mcmcseed2-phi10.0#3/img/#1} \\
    %
    {\tiny (g) Data1, MCMC3, $\phi=0$} &
    {\tiny (h) Data1, MCMC3, $\phi=1$} &
    {\tiny (i) Data1, MCMC3, $\phi=10$} &
    {\tiny (j) Data2, MCMC1, $\phi=0$} &
    {\tiny (k) Data2, MCMC1, $\phi=1$} &
    {\tiny (l) Data2, MCMC1, $\phi=10$} \\
    %
    \includegraphics[scale=#2]{\imgdir/dataseed1-mcmcseed3-phi0.0#3/img/#1} &
    \includegraphics[scale=#2]{\imgdir/dataseed1-mcmcseed3-phi1.0#3/img/#1} &
    \includegraphics[scale=#2]{\imgdir/dataseed1-mcmcseed3-phi10.0#3/img/#1} &
    %
    \includegraphics[scale=#2]{\imgdir/dataseed2-mcmcseed1-phi0.0#3/img/#1} &
    \includegraphics[scale=#2]{\imgdir/dataseed2-mcmcseed1-phi1.0#3/img/#1} &
    \includegraphics[scale=#2]{\imgdir/dataseed2-mcmcseed1-phi10.0#3/img/#1} \\
    %
    {\tiny (m) Data2, MCMC2, $\phi=0$} &
    {\tiny (n) Data2, MCMC2, $\phi=1$} &
    {\tiny (o) Data2, MCMC2, $\phi=10$} &
    {\tiny (p) Data2, MCMC3, $\phi=0$} &
    {\tiny (q) Data2, MCMC3, $\phi=1$} &
    {\tiny (r) Data2, MCMC3, $\phi=10$} \\
    %
    \includegraphics[scale=#2]{\imgdir/dataseed2-mcmcseed2-phi0.0#3/img/#1} &
    \includegraphics[scale=#2]{\imgdir/dataseed2-mcmcseed2-phi1.0#3/img/#1} &
    \includegraphics[scale=#2]{\imgdir/dataseed2-mcmcseed2-phi10.0#3/img/#1} &
    %
    \includegraphics[scale=#2]{\imgdir/dataseed2-mcmcseed3-phi0.0#3/img/#1} &
    \includegraphics[scale=#2]{\imgdir/dataseed2-mcmcseed3-phi1.0#3/img/#1} &
    \includegraphics[scale=#2]{\imgdir/dataseed2-mcmcseed3-phi10.0#3/img/#1} \\
    %
    {\tiny (s) Data3, MCMC1, $\phi=0$} &
    {\tiny (t) Data3, MCMC1, $\phi=1$} &
    {\tiny (u) Data3, MCMC1, $\phi=10$} &
    {\tiny (v) Data3, MCMC2, $\phi=0$} &
    {\tiny (w) Data3, MCMC2, $\phi=1$} &
    {\tiny (x) Data3, MCMC2, $\phi=10$} \\
    %
    \includegraphics[scale=#2]{\imgdir/dataseed3-mcmcseed1-phi0.0#3/img/#1} &
    \includegraphics[scale=#2]{\imgdir/dataseed3-mcmcseed1-phi1.0#3/img/#1} &
    \includegraphics[scale=#2]{\imgdir/dataseed3-mcmcseed1-phi10.0#3/img/#1} &
    %
    \includegraphics[scale=#2]{\imgdir/dataseed3-mcmcseed2-phi0.0#3/img/#1} &
    \includegraphics[scale=#2]{\imgdir/dataseed3-mcmcseed2-phi1.0#3/img/#1} &
    \includegraphics[scale=#2]{\imgdir/dataseed3-mcmcseed2-phi10.0#3/img/#1} \\
    %
    {\tiny (y) Data3, MCMC3, $\phi=0$} &
    {\tiny (z) Data3, MCMC3, $\phi=1$} &
    {\tiny ($\alpha$) Data3, MCMC3, $\phi=10$} &
    & & \\
    %
    \includegraphics[scale=#2]{\imgdir/dataseed3-mcmcseed3-phi0.0#3/img/#1} &
    \includegraphics[scale=#2]{\imgdir/dataseed3-mcmcseed3-phi1.0#3/img/#1} &
    \includegraphics[scale=#2]{\imgdir/dataseed3-mcmcseed3-phi10.0#3/img/#1} \\
    & & \\
  \end{tabular}
}


% Title Settings
\title{Simulation Study 6.7.20}
\author{Arthur Lui}
\date{\today} % \date{} to set date to empty

% MAIN %
\begin{document}

\maketitle

% \tableofcontents \newpage % Comment to remove table of contents

% \section{Objective}\label{sec:objective}
% TODO

\section{Data Generation}\label{sec:data-generation}
In this study, we generate data with two samples, each containing 2000 cells,
and 21 markers. Figure~\ref{fig:Z-true} and Table~\ref{tab:W-true}
respectively show the true $Z$ and $W$ used to generate the simulated data in
this study. Note the following:
\begin{itemize}
  \item Three $Z$'s, shown in Figure~\ref{fig:Z-true}, were used in this study.
  \begin{itemize}
    \item In $Z^{1,\true}$, each column is different from the other columns exactly
    by exactly two bits. In this sense, all columns are "close" to each
    other.
    \item In $Z^{2,\true}$, each column is different from the other columns exactly
    by exactly six bits. In this sense, all columns are "far" to each
    other.
    \item In $Z^{3,\true}$, the first three columns are similar. Columns 4 and 5 are 
    close to each other. And columns 6 and 7 are distinct from other columns. 
  \end{itemize}
  \item Figure~\ref{tab:W-true} shows the true $W$ used for this study. It
    contains feature abundances that are small in both samples, large in both
    samples, and large in only one sample. For example, feature $k=7$, 
    has an abundance of $3\%$ in sample 1, and is absent in feature 2. 
    Moreover, it has the smallest aggregate abundance across both samples. 
    Hence, we can expect it will be least likely recovered in the analysis. 
    Feature $k=3$, on the other hand is abundant in both samples, so we expect
    it to be easily recovered.
\end{itemize}

\begin{figure}[H]
  \begin{center}  % 6 x 5
    \begin{tabular}{ccc}
      \includegraphics[scale=0.4]{\imgdir/dataseed1-mcmcseed1-phi1.0-Zind1/img/Z_true.pdf} &
      \includegraphics[scale=0.4]{\imgdir/dataseed1-mcmcseed1-phi1.0-Zind2/img/Z_true.pdf} &
      \includegraphics[scale=0.4]{\imgdir/dataseed1-mcmcseed1-phi1.0-Zind3/img/Z_true.pdf} \\
      %
      (a) $Z^{1,\true}$ &
      (b) $Z^{2,\true}$ &
      (c) $Z^{3,\true}$ \\
      %
      column-distances of 2 bits &
      column-distances of 6 bits &
      mix of similar and distinct columns \\
    \end{tabular}
  \end{center}
  \caption{Simulation truth of $Z$ for three scenarios.}
  \label{fig:Z-true}
\end{figure}

% latex table generated in R 3.4.4 by xtable 1.8-3 package
% Wed Feb 12 14:53:27 2020
\begin{table}[ht]
  \centering
  \begin{tabular}{rrrrrrrr}
    \hline
    & $k=1$ & $k=2$ & $k=3$ & $k=4$ & $k=5$ & $k=6$ & $k=7$ \\
    \hline
    Sample 1 & 0.04 & 0.05 & 0.39 & 0.00 & 0.06 & 0.43 & 0.03 \\
    Sample 2 & 0.00 & 0.05 & 0.54 & 0.20 & 0.14 & 0.07 & 0.00 \\
    \hline
  \end{tabular}
  \caption{$W^\true$}
  \label{tab:W-true}
\end{table}

\subsection{More on data generation / MCMC setup}
\begin{itemize}
  \item We generated $y_{inj}$ from skew Normal distributions. Specifically,
  \begin{itemize}
    \item Location parameter: $(\mu_{0,j}^\star)^\true=-1 + \epsilon_j$, $\epsilon_j \sim \Uniform(-0.3, 0.3)$
    \item Location parameter: $(\mu_{1,j}^\star)^\true=1 + \epsilon_j$, $\epsilon_j \sim \Uniform(-0.3, 0.3)$
    \item Scale parameter: ($\sigma^2_i)^\true=0.5$, $N=(2000, 2000)$
    \item The skewness parameter $\zeta$ was set to -0.9. Note that $\zeta$
          corresponds to $\delta \in \p{-1, 1}$ in \cite{fruhwirth2010bayesian}.
  \end{itemize}
  \item 6000 burn-in, followed by 3000 iterations (no thinning).
  \item Computation time for each chains was was approximately 12.6 hours.
  \item \textbf{Some Priors / setup:}
  \begin{itemize}
    \item $w^\star_{i,k} \sim \Gam(\text{shape}=1, \text{rate}=1/2)$
    \item $p_c \sim \text{Beta}(1, 9)$
    \item $L_0=L_1=3$
    \item $\alpha \sim \Gam(0.1, 0.1)$
    \item $\delta_0 \sim \TN^-(1, 0.1)$
    \item $\delta_1 \sim \TN^+(1, 0.1)$
    % \item $\eta_{z, i,j,\ell} \sim \Dir_{L_z}(1)$
    % \item $\sigma^2_i \sim \InvGamma(3, 1)$
    \item $\phi \in \bc{0, 1, 10}$
    \item The simulation study was done for each of the true $Z$'s in 
          Figure~\ref{fig:Z-true}.
    \item Three chains (different random seeds) were run for each $\phi$ and data set.
    \item 5\% of data in each sample are used to sample from trained prior, and
      $M=5$ was used.
    \item This dataset contains missing data
  \end{itemize}
\end{itemize}

\newpage
\section{Log likelihood (post-burn)}
\begin{figure}[H]
  \begin{center}  % 6 x 5
    \imgtemplate{loglike_postburn.pdf}{.15}
  \end{center}
  \caption{Log-likelihood for 3000 iterations after burn-in of 6000 for each
  simulation setup. \textbf{More burn-in} is needed in at least three of the runs
  (e.g. (k), (l), (t)).}
  \label{fig:ll}
\end{figure}

\newpage
\section{Posterior mean of $Z$}
\begin{figure}[H]
  \begin{center}  % 6 x 5
    \imgtemplate{Zmean.pdf}{0.15}
  \end{center}
  % TODO: Write more?
  \caption{Posterior means of $Z$ for each run. When $\phi=0$, the number of
  recovered features is 0 to 1, for $Z^{1,\true}$. In contrast, number of
  recovered features when $\phi > 0$ is between 6 and 7, with the missing
  feature being feature 7. This is perhaps due to the added left-skew in the observed
  data, making it more feasible for features to be explained by mixture components
  with negative means. The $\sigma_2^i$ learned when $\phi=0$ tended to be
  much larger as well. Again, mixing may also be better when $\phi>0$ because 
  the exploration of different configurations of $Z$ is encourage. There is a
  tendency for $Z$ to be extremely sparse when $\phi=0$, due to the IBP prior.
  %
  For $Z^{2,\true}$, between 2 and 6 features were recovered when $\phi=0$.
  When $\phi > 0$, all features were recovered. But in $(n)$, 
  eight features were recovered, the extra feature being similar to one of the
  features. However, it's estimated abundance was less than 1\%.
  %
  For Data 3, when $\phi=0$, the only features that were recovered are
  features 3, 5, 6, and 7, which are the most representative features. i.e.
  Features 1 and 2, which are rare, are merged with feature 3, which is
  abundant and similar to the former features. When $\phi>0$, 6 to 7 features were
  recovered. The missing features were either feature 1 or 2, which is similar to 
  and merged with feature 3. Feature 7, though rare, was recovered, likely because
  it is distinct.}
  \label{fig:zmean}
\end{figure}
 
\newpage
\section{Posterior estimate of $Z_1$}
\begin{figure}[H]
  \begin{center}  % 6 x 5
    \imgtemplate{Z1.pdf}{0.15}
  \end{center}
  % TODO: Write more?
  \caption{}
  \label{fig:z1est}
\end{figure}

\newpage
\section{Posterior estimate of $Z_2$}
\begin{figure}[H]
  \begin{center}  % 6 x 5
    \imgtemplate{Z2.pdf}{0.15}
  \end{center}
  % TODO: Write more?
  \caption{}
  \label{fig:z2est}
\end{figure}

\newpage
\section{Posterior estimate of $y_1$}
\begin{figure}[H]
  \begin{center}  % 6 x 5
    \imgtemplate{y1.pdf}{0.15}
  \end{center}
  % TODO: Write more?
  \caption{}
  \label{fig:y1est}
\end{figure}

\newpage
\section{Posterior estimate of $y_2$}
\begin{figure}[H]
  \begin{center}  % 6 x 5
    \imgtemplate{y2.pdf}{0.15}
  \end{center}
  % TODO: Write more?
  \caption{}
  \label{fig:y2est}
\end{figure}

\newpage
\section{Box plots for $p_i$}
\begin{figure}[H]
  \begin{center}  % 6 x 5
    \imgtemplate{p.pdf}{0.15}
  \end{center}
  % TODO: Write more?
  \caption{The posterior mean of $p_i$ is usually between 0.2 and 0.4, and
  sometimes between 0.4 and 0.5. The spread is large.}
  \label{fig:ppost}
\end{figure}

\newpage
\section{Trace plots for $p_1$}
\begin{figure}[H]
  \begin{center}  % 6 x 5
    \imgtemplate{p1_trace.pdf}{0.15}
  \end{center}
  % TODO: Write more?
  \caption{}
  \label{fig:p1trace}
\end{figure}

\newpage
\section{Trace plots for $p_2$}
\begin{figure}[H]
  \begin{center}  % 6 x 5
    \imgtemplate{p2_trace.pdf}{0.15}
  \end{center}
  % TODO: Write more?
  \caption{}
  \label{fig:p2trace}
\end{figure}

\newpage
\section{Trace plots for $\mu^\star$}
\begin{figure}[H]
  \begin{center}  % 6 x 5
    \imgtemplate{mus_trace.pdf}{0.15}
  \end{center}
  % TODO: Write more?
  \caption{Trace plots for $\mu^\star_z$. Note that in (a) and (j) the variance
  for the positive mean components is large because they are not used. Effectively,
  they are being sampled from the prior.}
  \label{fig:mus-trace}
\end{figure}

\newpage
\section{Posterior distributions for $\mu^\star$}
\begin{figure}[H]
  \begin{center}  % 6 x 5
    \imgtemplate{mus.pdf}{0.15}
  \end{center}
  % TODO: Write more?
  \caption{}
  \label{fig:mus}
\end{figure}


\newpage
\section{Trace plots for $\sigma^2$}
\begin{figure}[H]
  \begin{center}  % 6 x 5
    \imgtemplate{sig2_trace.pdf}{0.15}
  \end{center}
  % TODO: Write more?
  \caption{Trace plots for $\sigma^2_i$. Note the jump in (t) is similar
  to the jump in the log likelihood.}
  \label{fig:sig2-trace}
\end{figure}

\newpage
\section{Posterior distributions for $\sigma^2$}
\begin{figure}[H]
  \begin{center}  % 6 x 5
    \imgtemplate{sig2.pdf}{0.15}
  \end{center}
  % TODO: Write more?
  \caption{}
  \label{fig:sig2}
\end{figure}

\newpage
\section{Posterior distributions for sample 1 marker 1}
\begin{figure}[H]
  \begin{center}  % 6 x 5
    \imgtemplate{dden/dden_i1_j1.pdf}{0.15}
  \end{center}
  % TODO: Write more?
  \caption{Data fit is rather good.}
  \label{fig:ddi1j1}
\end{figure}

\newpage
\section{Posterior distributions for sample 1 marker 2}
\begin{figure}[H]
  \begin{center}  % 6 x 5
    \imgtemplate{dden/dden_i1_j2.pdf}{0.15}
  \end{center}
  % TODO: Write more?
  \caption{}
  \label{fig:ddi1j2}
\end{figure}

\newpage
\section{Posterior distributions for sample 2 marker 1}
\begin{figure}[H]
  \begin{center}  % 6 x 5
    \imgtemplate{dden/dden_i2_j1.pdf}{0.15}
  \end{center}
  % TODO: Write more?
  \caption{}
  \label{fig:ddi2j1}
\end{figure}

\newpage
\section{Posterior distributions for sample 2 marker 2}
\begin{figure}[H]
  \begin{center}  % 6 x 5
    \imgtemplate{dden/dden_i2_j2.pdf}{0.15}
  \end{center}
  % TODO: Write more?
  \caption{}
  \label{fig:ddi2j2}
\end{figure}

\newpage
\section{Posterior distributions for sample 2 marker 6}
\begin{figure}[H]
  \begin{center}  % 6 x 5
    \imgtemplate{dden/dden_i1_j6.pdf}{0.15}
  \end{center}
  % TODO: Write more?
  \caption{}
  \label{fig:ddi1j6}
\end{figure}

\newpage
\section{Posterior distributions for sample 2 marker 6}
\begin{figure}[H]
  \begin{center}  % 6 x 5
    \imgtemplate{dden/dden_i2_j6.pdf}{0.15}
  \end{center}
  % TODO: Write more?
  \caption{}
  \label{fig:ddi2j6}
\end{figure}

\newpage
\section{Missing Mechanism}
\begin{figure}[H]
  \begin{center}  % 6 x 5
    \begin{tabular}{cc}
      \includegraphics[scale=.5]{\imgdir/dataseed1-mcmcseed1-phi1.0-Zind1/img/missmech_1.pdf} &
      \includegraphics[scale=.5]{\imgdir/dataseed1-mcmcseed1-phi1.0-Zind1/img/missmech_2.pdf} \\
      \includegraphics[scale=.5]{\imgdir/dataseed1-mcmcseed1-phi1.0-Zind2/img/missmech_1.pdf} &
      \includegraphics[scale=.5]{\imgdir/dataseed1-mcmcseed1-phi1.0-Zind2/img/missmech_2.pdf} \\
      \includegraphics[scale=.5]{\imgdir/dataseed1-mcmcseed1-phi1.0-Zind3/img/missmech_1.pdf} &
      \includegraphics[scale=.5]{\imgdir/dataseed1-mcmcseed1-phi1.0-Zind3/img/missmech_2.pdf} \\
    \end{tabular}
  \end{center}
\label{fig:missmech}
\end{figure}

% Uncomment if using bibliography:
\bibliography{sim}

\end{document}
