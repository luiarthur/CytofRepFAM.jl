\documentclass[10pt]{article} % 12-point font

\usepackage[margin=1.5cm]{geometry} % set page to 1-inch margins
\usepackage{amsmath} % for math
\usepackage{amssymb} % like \Rightarrow
\setlength\parindent{0pt} % Suppresses the indentation of new paragraphs.

% Big display
\newcommand{\ds}{ \displaystyle }
% Parenthesis
\newcommand{\norm}[1]{\left\lVert#1\right\rVert}
\newcommand{\p}[1]{\left(#1\right)}
\newcommand{\bk}[1]{\left[#1\right]}
\newcommand{\bc}[1]{ \left\{#1\right\} }
\newcommand{\abs}[1]{ \left|#1\right| }
% Derivatives
\newcommand{\df}[2]{ \frac{d#1}{d#2} }
\newcommand{\ddf}[2]{ \frac{d^2#1}{d{#2}^2} }
\newcommand{\pd}[2]{ \frac{\partial#1}{\partial#2} }
\newcommand{\pdd}[2]{\frac{\partial^2#1}{\partial{#2}^2} }
% Distributions
\newcommand{\Normal}{ \text{Normal} }
\newcommand{\Beta}{ \text{Beta} }
\newcommand{\Gam}{ \text{Gamma} }
\newcommand{\InvGamma}{ \text{Inv-Gamma} }
\newcommand{\Uniform}{ \text{Uniform} }
\def\Dir{\text{Dirichlet}}
\def\TN{\text{TN}}
% Statistics
\newcommand{\E}{\text{E}}
\newcommand{\iid}{\overset{iid}{\sim}}
\newcommand{\ind}{\overset{ind}{\sim}}

% Graphics
\usepackage{graphicx}  % for figures
\usepackage{float} % Put figure exactly where I want [H]

% Uncomment if using bibliography
% Bibliography
% \usepackage{natbib}
% \bibliographystyle{plainnat}

% Adds settings for hyperlinks. (Mainly for table of contents.)
\usepackage{hyperref}
\hypersetup{
  pdfborder={0 0 0} % removes red box from links
}


% Macros for this project
\def\true{\text{TRUE}}
\usepackage[dvipsnames,usenames]{color}
\newcommand{\bbh}{\color{blue}\textbf}  % blue bold highlight
\newcommand{\bch}{\color{blue}\it}  % blue italics
\newcommand{\ech}{\color{black}\rm}
% \usepackage{listings}
\def\imgdir{../../results/test-sim-6-6/}
\newcommand{\imgtemplate}[3]{
  \begin{tabular}{cccccc}
    {\tiny (a) Data1, MCMC1, $\phi=0$} &
    {\tiny (b) Data1, MCMC1, $\phi=1$}&
    {\tiny (c) Data1, MCMC1, $\phi=10$} &
    {\tiny (d) Data1, MCMC2, $\phi=0$} &
    {\tiny (e) Data1, MCMC2, $\phi=1$} &
    {\tiny (f) Data1, MCMC2, $\phi=10$} \\
    %
    \includegraphics[scale=#2]{\imgdir/dataseed1-mcmcseed1-phi0.0#3/img/#1} &
    \includegraphics[scale=#2]{\imgdir/dataseed1-mcmcseed1-phi1.0#3/img/#1} &
    \includegraphics[scale=#2]{\imgdir/dataseed1-mcmcseed1-phi10.0#3/img/#1} &
    %
    \includegraphics[scale=#2]{\imgdir/dataseed1-mcmcseed2-phi0.0#3/img/#1} &
    \includegraphics[scale=#2]{\imgdir/dataseed1-mcmcseed2-phi1.0#3/img/#1} &
    \includegraphics[scale=#2]{\imgdir/dataseed1-mcmcseed2-phi10.0#3/img/#1} \\
    %
    {\tiny (g) Data1, MCMC3, $\phi=0$} &
    {\tiny (h) Data1, MCMC3, $\phi=1$} &
    {\tiny (i) Data1, MCMC3, $\phi=10$} &
    {\tiny (j) Data2, MCMC1, $\phi=0$} &
    {\tiny (k) Data2, MCMC1, $\phi=1$} &
    {\tiny (l) Data2, MCMC1, $\phi=10$} \\
    %
    \includegraphics[scale=#2]{\imgdir/dataseed1-mcmcseed3-phi0.0#3/img/#1} &
    \includegraphics[scale=#2]{\imgdir/dataseed1-mcmcseed3-phi1.0#3/img/#1} &
    \includegraphics[scale=#2]{\imgdir/dataseed1-mcmcseed3-phi10.0#3/img/#1} &
    %
    \includegraphics[scale=#2]{\imgdir/dataseed2-mcmcseed1-phi0.0#3/img/#1} &
    \includegraphics[scale=#2]{\imgdir/dataseed2-mcmcseed1-phi1.0#3/img/#1} &
    \includegraphics[scale=#2]{\imgdir/dataseed2-mcmcseed1-phi10.0#3/img/#1} \\
    %
    {\tiny (m) Data2, MCMC2, $\phi=0$} &
    {\tiny (n) Data2, MCMC2, $\phi=1$} &
    {\tiny (o) Data2, MCMC2, $\phi=10$} &
    {\tiny (p) Data2, MCMC3, $\phi=0$} &
    {\tiny (q) Data2, MCMC3, $\phi=1$} &
    {\tiny (r) Data2, MCMC3, $\phi=10$} \\
    %
    \includegraphics[scale=#2]{\imgdir/dataseed2-mcmcseed2-phi0.0#3/img/#1} &
    \includegraphics[scale=#2]{\imgdir/dataseed2-mcmcseed2-phi1.0#3/img/#1} &
    \includegraphics[scale=#2]{\imgdir/dataseed2-mcmcseed2-phi10.0#3/img/#1} &
    %
    \includegraphics[scale=#2]{\imgdir/dataseed2-mcmcseed3-phi0.0#3/img/#1} &
    \includegraphics[scale=#2]{\imgdir/dataseed2-mcmcseed3-phi1.0#3/img/#1} &
    \includegraphics[scale=#2]{\imgdir/dataseed2-mcmcseed3-phi10.0#3/img/#1} \\
    %
    {\tiny (s) Data3, MCMC1, $\phi=0$} &
    {\tiny (t) Data3, MCMC1, $\phi=1$} &
    {\tiny (u) Data3, MCMC1, $\phi=10$} &
    {\tiny (v) Data3, MCMC2, $\phi=0$} &
    {\tiny (w) Data3, MCMC2, $\phi=1$} &
    {\tiny (x) Data3, MCMC2, $\phi=10$} \\
    %
    \includegraphics[scale=#2]{\imgdir/dataseed3-mcmcseed1-phi0.0#3/img/#1} &
    \includegraphics[scale=#2]{\imgdir/dataseed3-mcmcseed1-phi1.0#3/img/#1} &
    \includegraphics[scale=#2]{\imgdir/dataseed3-mcmcseed1-phi10.0#3/img/#1} &
    %
    \includegraphics[scale=#2]{\imgdir/dataseed3-mcmcseed2-phi0.0#3/img/#1} &
    \includegraphics[scale=#2]{\imgdir/dataseed3-mcmcseed2-phi1.0#3/img/#1} &
    \includegraphics[scale=#2]{\imgdir/dataseed3-mcmcseed2-phi10.0#3/img/#1} \\
    %
    {\tiny (y) Data3, MCMC3, $\phi=0$} &
    {\tiny (z) Data3, MCMC3, $\phi=1$} &
    {\tiny ($\alpha$) Data3, MCMC3, $\phi=10$} &
    & & \\
    %
    \includegraphics[scale=#2]{\imgdir/dataseed3-mcmcseed3-phi0.0#3/img/#1} &
    \includegraphics[scale=#2]{\imgdir/dataseed3-mcmcseed3-phi1.0#3/img/#1} &
    \includegraphics[scale=#2]{\imgdir/dataseed3-mcmcseed3-phi10.0#3/img/#1} \\
    & & \\
  \end{tabular}
}


% Title Settings
\title{Simulation Study 6.6}
\author{Arthur Lui}
\date{\today} % \date{} to set date to empty

% MAIN %
\begin{document}

\maketitle

% \tableofcontents \newpage % Comment to remove table of contents

% \section{Objective}\label{sec:objective}
% TODO

\section{Data Generation}\label{sec:data-generation}
Figure~\ref{fig:Z-true} and Table~\ref{tab:W-true} respectively show the true
$Z$ and $W$ used to generate the simulated data in this study. Note the following:
\begin{itemize}
  \item Features (columns) 1, 2, and 3 are similar (maximum difference of 2).
    In Sample 1, the first two features are rare / absent in one sample, and
    the third feature is abundant in both samples. Previously, we wanted to
    know if the first two features are prone to being absorbed into feature 3. We
    found that using the prior $w^\star_{i,k} \sim \Gam(10, 1)$, and using priors
    that encouraged $r_{i,k}$ to be sparse, Features 1 and 2 tend to merge
    (into Feature 2, which is slightly larger), and Feature 3 would be in its
    own cluster or split into two smaller identical features. We believe this is 
    due to prior information in $w^\star$ encouraging features to be equally weighted.
    In this study, we continue to monitor these kinds of behaviour.
  \item Features 4 and 5 are similar (difference of 1). They are rare / absent
    in one sample and abundant in another. Previously, we were also curious if
    these two features will be merged. Although, we didn't expect them to. It was
    conceivable that they will only be ``selected'' in one sample, but not in
    another. We found that they were always recovered, in part, we believe, because
    they had a large presence in one sample. We continue to monitor their behaviour
    here.
  \item Feature 6 is abundant in Sample 1 but not in Sample 2. We expect it to be
    recovered in $Z$ but it is conceivable that it will not be ``selected'' in sample
    2. It is distinct from other features.
  \item Feature 7 is rare in Sample 1, and absent in Sample 2. Nevertheless, we
    expect it to recovered in $Z$ because it is different from the other
    features.
\end{itemize}

\begin{figure}[H]
  \begin{center}  % 6 x 5
    \includegraphics[scale=.4]{\imgdir/dataseed_1/mcmcseed_1/scale_0/img/Z_true.pdf}
  \end{center}
  \caption{$Z^\true$}
  \label{fig:Z-true}
\end{figure}

% latex table generated in R 3.4.4 by xtable 1.8-3 package
% Wed Feb 12 14:53:27 2020
\begin{table}[ht]
  \centering
  \begin{tabular}{rrrrrrrr}
    \hline
    & $k=1$ & $k=2$ & $k=3$ & $k=4$ & $k=5$ & $k=6$ & $k=7$ \\
    \hline
    Sample 1 & 0.04 & 0.05 & 0.39 & 0.00 & 0.06 & 0.43 & 0.03 \\
    Sample 2 & 0.00 & 0.05 & 0.54 & 0.20 & 0.14 & 0.07 & 0.00 \\
    \hline
  \end{tabular}
  \caption{$W^\true$}
  \label{tab:W-true}
\end{table}

\subsection{More on data generation / MCMC setup}
\begin{itemize}
  \item $(\mu_{0,j}^\star)^\true=-2 + \epsilon_j$, $\epsilon_j \sim \Uniform(-0.5, 0.5)$
  \item $(\mu_{1,j}^\star)^\true=2 + \epsilon_j$, $\epsilon_j \sim \Uniform(-0.5, 0.5)$
  \item ($\sigma^2_i)^\true=0.5$, $N=(2000, 2000)$
  \item 10000 burn-in, followed by 4000 iterations, and every other sample was
    kept (2000 samples total).
  \item Three datasets were generated using the true parameters. Three MCMC runs using 
    different random seeds were done. For each run, $\phi$ was set at (0, 1, 10). There
    are a total of 27 runs.
  \item \textbf{Some Priors:}
  \begin{itemize}
    \item $w^\star_{i,k} \sim \Gam(\text{shape}=10, \text{rate}=1)$
    \item $\eta_{z, i,j,\ell} \sim \Dir_{L_z}(5)$
    \item $\alpha \sim \Gam(0.1, 0.1)$
    \item $\delta_0 \sim \TN^-(1.0, 0.1)$
    \item $\delta_1 \sim \TN^+(1.0, 0.1)$
    \item $\sigma^2_i \sim \InvGamma^+(1.0, 0.1)$
  \end{itemize}
\end{itemize}

\newpage
\section{Log likelihood (post-burn)}
\begin{figure}[H]
  \begin{center}  % 6 x 5
    \imgtemplate{loglike_postburn.pdf}{.15}
  \end{center}
  \caption{Post burn-in log likelihoods for each run. Burn in used was 10000.
    These plots show the log-likelihood evaluations of the 4000 subsequent
    iterations for each simulated dataset, each replicate, and each $\phi$.
    Unlike in the previous simulation study, we see no clear evidence that 
    the chains have not converged. However, from Table~\ref{tab:metrics}, it's
    clear that some chains are stuck in local modes.}
  \label{fig:ll}
\end{figure}

\newpage
\section{Posterior mean of $Z$}
\begin{figure}[H]
  \begin{center}  % 6 x 5
    \imgtemplate{Zmean.pdf}{.2}
  \end{center}
  \caption{Recall that in the simulation truth, 6 and 5 features are present in
    samples 1 and 2 respectively. Consequently, 7 features should be active
    (i.e. selected in at least one sample) in each figure (indicated by high
    posterior probability). In some of these scenarios, only 6 features are
    active, in others 8 are selected. There are also cases were only 5 features
    are active. But most of the time, 7 are selected.  Eight features are
    selected only when $\phi=0$. That is, one feature is duplicated.  The
    feature that is duplicated is feature 3 (from Figure~\ref{fig:Z-true}.) It
    can be seen in Figures~(\ref{fig:z1}) and (\ref{fig:z2}), that feature 3 is
    split rather evenly into two identical features. In the cases where only 6
    features are selected, features 1 and 2 were merged into one group. I
    believe the merging of small, similar features, and the splitting of large
    features occurs because the prior for $W$ encourages weight to be spread
    evenly between features. Moreover, in the case where only 5 features are
    active, features 1 and 2 were merged into feature 3, which is a similar but
    much more abundant feature (see (r), (w)).}
  \label{fig:zmean}
\end{figure}

\newpage
\section{Posterior estimate of $Z_1$}
\begin{figure}[H]
  \begin{center}  % 6 x 5
    \imgtemplate{Z1.pdf}{.17}
  \end{center}
  \label{fig:z1}
\end{figure}

\newpage
\section{Posterior estimate of $Z_2$}
\begin{figure}[H]
  \begin{center}  % 6 x 5
    \imgtemplate{Z2.pdf}{.17}
  \end{center}
  \label{fig:z2}
\end{figure}

\newpage
\section{Trace plots of $\mu^\star$}
\begin{figure}[H]
  \begin{center}  % 6 x 5
    \imgtemplate{mus_trace.pdf}{.2}
  \end{center}
  \label{fig:mutrace}
  \caption{Notice here that for (r) and (w), $\mu$ is near 0.}
\end{figure}

\newpage
\section{Posterior distributions of $\mu^\star$}
\begin{figure}[H]
  \begin{center}  % 6 x 5
    \imgtemplate{mus.pdf}{.2}
  \end{center}
  \label{fig:mu}
\end{figure}

\newpage
\section{Trace plots of $\sigma^2$}
\begin{figure}[H]
  \begin{center}  % 6 x 5
    \imgtemplate{sig2_trace.pdf}{.2}
  \end{center}
  \label{fig:sigtrace}
\end{figure}

\newpage
\section{Posterior distribution of $\sigma^2$}
\begin{figure}[H]
  \begin{center}  % 6 x 5
    \imgtemplate{sig2.pdf}{.2}
  \end{center}
  \label{fig:sig}
\end{figure}

\newpage
\section{Posterior distribution of $W$}
\begin{figure}[H]
  \begin{center}  % 6 x 5
    \imgtemplate{W.pdf}{.2}
  \end{center}
  \label{fig:w}
\end{figure}

% \newpage
% \section{Posterior distribution of $v$}
% \begin{figure}[H]
%   \begin{center}  % 6 x 5
%     \imgtemplate{v.pdf}{.2}
%   \end{center}
% \end{figure}

\newpage
\section{Posterior distribution of $\lambda_1$}
\begin{figure}[H]
  \begin{center}  % 6 x 5
    \imgtemplate{y1.pdf}{.2}
  \end{center}
  % \caption{Notice here, the large groups are split into two groups (e.g.  (j),
  % (s), (p)). Small and similar groups are grouped together (e.g.  (h),
  % ($\alpha$)).}
  \label{fig:lam1}
\end{figure}

\newpage
\section{Posterior distribution of $\lambda_2$}
\begin{figure}[H]
  \begin{center}  % 6 x 5
    \imgtemplate{y2.pdf}{.2}
  \end{center}
  % \caption{Notice here, the large groups are split into two groups (e.g.  (a),
  % (g), (j), (p), (s)).}
  \label{fig:lam2}
\end{figure}

\newpage
\section{Posterior predictive density for sample 1, marker 1}
\begin{figure}[H]
  \begin{center}  % 6 x 5
    \imgtemplate{dden/dden_i1_j1.pdf}{.2}
  \end{center}
  \label{fig:dd11}
  \caption{Strange things in (r), (w). Seems to be mixing issues, getting trapped in local modes.}
\end{figure}

\newpage
\section{Posterior predictive density for sample 1, marker 2}
\begin{figure}[H]
  \begin{center}  % 6 x 5
    \imgtemplate{dden/dden_i1_j2.pdf}{.2}
  \end{center}
  \label{fig:dd12}
\end{figure}

\newpage
\section{Posterior predictive density for sample 2, marker 1}
\begin{figure}[H]
  \begin{center}  % 6 x 5
    \imgtemplate{dden/dden_i2_j1.pdf}{.2}
  \end{center}
  \label{fig:dd21}
  % \caption{Strange things in (h), (i), and (z)?}
\end{figure}

\newpage
\section{Posterior predictive density for sample 2, marker 2}
\begin{figure}[H]
  \begin{center}  % 6 x 5
    \imgtemplate{dden/dden_i2_j2.pdf}{.2}
  \end{center}
  \label{fig:dd22}
\end{figure}

\newpage
\section{Metrics}
% latex table generated in R 3.4.4 by xtable 1.8-3 package
% Thu Feb 13 22:57:36 2020
\begin{table}[H]
\centering
\begin{tabular}{crrrrrrrrrrr}
  \hline
  \rule{0pt}{2.6ex}
  & Data & MCMC & $\phi$ & DIC & LPML & $\hat{R}_1$ & $\hat{R}_2$& $R_1 (2.5\%)$ & $R_2 (2.5\%)$ & $R_1 (97.5\%)$ & $R_2 (97.5\%)$ \\
  \hline
  a) &1 &1 & 0 & -781118.53 & -18.22 & 5.15 & 5.77 & 5.00 & 5.00 & 6.00 & 6.00 \\
  b) &1 &1 & 1 & -780253.61 & -18.20 & 5.00 & 5.00 & 5.00 & 5.00 & 5.00 & 5.00 \\
  c) &1 &1 &10 & -878131.60 & -17.72 & 6.00 & 5.00 & 6.00 & 5.00 & 6.00 & 5.00 \\
  \hline                                           
  d) &1 &2 & 0 & -877944.78 & -17.75 & 6.00 & 5.00 & 6.00 & 5.00 & 6.00 & 5.00 \\
  e) &1 &2 & 1 & -879072.72 & -17.73 & 6.00 & 5.00 & 6.00 & 5.00 & 6.00 & 5.00 \\
  f) &1 &2 &10 & -780334.99 & -18.21 & 5.00 & 5.00 & 5.00 & 5.00 & 5.00 & 5.00 \\
  \hline                                           
  g) &1 &3 & 0 & -781079.80 & -18.21 & 5.00 & 5.00 & 5.00 & 5.00 & 5.00 & 5.00 \\
  h) &1 &3 & 1 & -780999.12 & -18.20 & 5.00 & 5.00 & 5.00 & 5.00 & 5.00 & 5.00 \\
  i) &1 &3 &10 & -878139.99 & -17.75 & 6.00 & 5.00 & 6.00 & 5.00 & 6.00 & 5.00 \\
  \hline\hline                                     
  j) &2 &1 & 0 & -861980.28 & -18.15 & 5.00 & 5.00 & 5.00 & 5.00 & 5.00 & 5.00 \\
  k) &2 &1 & 1 & -793929.95 & -18.48 & 5.00 & 4.00 & 5.00 & 4.00 & 5.00 & 4.00 \\
  l) &2 &1 &10 & -861537.43 & -18.18 & 5.00 & 5.00 & 5.00 & 5.00 & 5.00 & 5.00 \\
  \hline                                           
  m) &2 &2 & 0 & -921424.46 & -17.87 & 6.99 & 6.00 & 7.00 & 6.00 & 7.00 & 6.00 \\
  n) &2 &2 & 1 & -794848.92 & -18.47 & 5.00 & 4.00 & 5.00 & 4.00 & 5.00 & 4.00 \\
  o) &2 &2 &10 & -921825.37 & -17.88 & 6.00 & 5.00 & 6.00 & 5.00 & 6.00 & 5.00 \\
  \hline                                           
  p) &2 &3 & 0 & -861615.02 & -18.18 & 5.04 & 6.00 & 5.00 & 6.00 & 6.00 & 6.00 \\
  q) &2 &3 & 1 & -794300.71 & -18.47 & 5.00 & 4.00 & 5.00 & 4.00 & 5.00 & 4.00 \\
  r) &2 &3 &10 & -682542.83 & -18.87 & 4.00 & 4.00 & 4.00 & 4.00 & 4.00 & 4.00 \\
  \hline\hline                                     
  s) &3 &1 & 0 & -763241.45 & -18.20 & 6.00 & 5.00 & 6.00 & 5.00 & 6.00 & 5.00 \\
  t) &3 &1 & 1 & -763658.98 & -18.21 & 5.00 & 5.00 & 5.00 & 5.00 & 5.00 & 5.00 \\
  u) &3 &1 &10 & -676237.65 & -18.61 & 5.00 & 4.00 & 5.00 & 4.00 & 5.00 & 4.00 \\
  \hline                                           
  v) &3 &2 & 0 & -851310.67 & -17.79 & 7.00 & 6.00 & 7.00 & 6.00 & 7.00 & 6.00 \\
  w) &3 &2 & 1 & -557653.90 & -19.06 & 4.00 & 4.00 & 4.00 & 4.00 & 4.00 & 4.00 \\
  x) &3 &2 &10 & -851130.61 & -17.80 & 6.00 & 5.00 & 6.00 & 5.00 & 6.00 & 5.00 \\
  \hline                                           
  y) &3 &3 & 0 & -764177.86 & -18.21 & 6.00 & 6.00 & 6.00 & 6.00 & 6.00 & 6.00 \\
  z) &3 &3 & 1 & -763702.20 & -18.21 & 5.00 & 5.00 & 5.00 & 5.00 & 5.00 & 5.00 \\
  $\alpha$) &3 &3 &10 & -851580.44 & -17.78 & 6.00 & 5.00 & 6.00 & 5.00 & 6.00 & 5.00 \\
  \hline
\end{tabular}
\caption{Metrics}
\label{tab:metrics}
\end{table}

\newpage
\section{Missing Mechanism}
TODO: Include missing mechanisms


% Uncomment if using bibliography:
% \bibliography{bib}
\end{document}
