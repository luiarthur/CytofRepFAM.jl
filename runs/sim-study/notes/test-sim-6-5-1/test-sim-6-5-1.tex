\documentclass[12pt]{article} % 12-point font

\usepackage[margin=1in]{geometry} % set page to 1-inch margins
\usepackage{amsmath} % for math
\usepackage{amssymb} % like \Rightarrow
\setlength\parindent{0pt} % Suppresses the indentation of new paragraphs.

% Big display
\newcommand{\ds}{ \displaystyle }
% Parenthesis
\newcommand{\norm}[1]{\left\lVert#1\right\rVert}
\newcommand{\p}[1]{\left(#1\right)}
\newcommand{\bk}[1]{\left[#1\right]}
\newcommand{\bc}[1]{ \left\{#1\right\} }
\newcommand{\abs}[1]{ \left|#1\right| }
% Derivatives
\newcommand{\df}[2]{ \frac{d#1}{d#2} }
\newcommand{\ddf}[2]{ \frac{d^2#1}{d{#2}^2} }
\newcommand{\pd}[2]{ \frac{\partial#1}{\partial#2} }
\newcommand{\pdd}[2]{\frac{\partial^2#1}{\partial{#2}^2} }
% Distributions
\newcommand{\Normal}{ \text{Normal} }
\newcommand{\Beta}{ \text{Beta} }
\newcommand{\G}{ \text{Gamma} }
\newcommand{\InvGamma}{ \text{Inv-Gamma} }
\newcommand{\Uniform}{ \text{Uniform} }
% Statistics
\newcommand{\E}{ \text{E} }
\newcommand{\iid}{\overset{iid}{\sim}}
\newcommand{\ind}{\overset{ind}{\sim}}

% Graphics
\usepackage{graphicx}  % for figures
\usepackage{float} % Put figure exactly where I want [H]

% Commands for this project
\def\true{\text{TRUE}}
\usepackage[dvipsnames,usenames]{color}
\newcommand{\bbh}{\color{blue}\textbf}  % blue bold highlight
\newcommand{\bch}{\color{blue}\it}
\newcommand{\ech}{\color{black}\rm}
\usepackage{listings}

% Uncomment if using bibliography
% Bibliography
% \usepackage{natbib}
% \bibliographystyle{plainnat}

% Adds settings for hyperlinks. (Mainly for table of contents.)
\usepackage{hyperref}
\hypersetup{
  pdfborder={0 0 0} % removes red box from links
}

% Title Settings
\title{Test Simulation 6.5.1}
\author{Arthur Lui}
\date{\today} % \date{} to set date to empty

% MAIN %
\begin{document}

\maketitle

\tableofcontents \newpage % Comment to remove table of contents

\section{Objective}\label{sec:objective}
In test-sim-6-3, we saw that the posterior uncertainty for $R_i$ was non-existent. 
Consequently, for this simulation study, we want to use a more complicated
$Z^\true$ and $w^\true_i$, and see whether that increases posterior uncertainty
for $R_i$.

\section{Data Generation}\label{sec:data-generation}
The true $Z$ used in this study is in Figure~\ref{fig:Z-true}.
Notice in the $Z$ matrix, 
\begin{itemize}
  \item columns 1 \& 2 are similar (difference of 1)
  \item columns 3 \& 4 (difference of 1)
  \item columns 5 - 7 are distinct columns
\end{itemize}

\begin{figure}[h]
  \begin{center}
    \includegraphics[scale=.5]{img/test-sim-6-5-1/seed_1/scale_0/Kmcmc_5/img/Z_true.pdf}
  \end{center}
  \label{fig:Z-true}
  \caption{True $Z$. Columns 1 and 2 are similar (differs by 1). Columns 3 and
  4 are also similar (differs by 1). Columns 5 to 7 are distinct.}
\end{figure}

The true $W$ used in this study is in Table~\ref{tab:W-true}.  In conjunction
with $Z^\true$, this $W^\true$, makes the first two (similar) features in sample 1
(first row) absent in the other sample. The 3rd and 4th features are rare, with
the 3rd feature being absent in the first sample. The remaining features make
up the majority.

\begin{table}[H]
  \centering
  \begin{tabular}{|c|rrrrrrr|}
    \hline
    Sample & $k=1$ & $k=2$ & $k=3$ & $k=4$ & $k=5$ & $k=6$ & $k=7$ \\
    \hline
    1 & 4\% & 5\% & {\bbh 0\%} & 6\% & 44\% & 38\% & 3\% \\
    2 & {\bbh 0\%} & {\bbh 0\%} & 6\% & 5\% & 7\% & 54\% & 28\% \\
    \hline
  \end{tabular}
  \caption{True $W$ used in simulation study.}
  \label{tab:W-true}
\end{table}

Other data simulation settings:
\begin{itemize}
  \item $N$ was set to (1000, 1000)
  \item $\mu_0^\star = -2$, and $\mu_1^\star =2$
  \item $\sigma_i^2 = 0.5$
  \item Three data sets were generated (using different random seeds).
\end{itemize}


\section{MCMC settings}\label{sec:mcmc-settings}
\begin{itemize}
  \item Burn-in period: 10000
  \item Number of MCMC samples collected: 2000 (thinned by 2)
  \item Model was fit for $K=\bc{5, 6, 7, 8, 15}$ and $\phi$ the scale used in
    the RepFAM penalty term = $\bc{0, 1, 10}$. These models were fit for each of the
    three generated data sets. So in total, 45 runs were done.
\end{itemize}

\section{Results}\label{sec:results}
Images generated from the results are zipped in the google drive at this link: \\
\url{https://drive.google.com/open?id=1-hBNgiD1usNvI7WxQ-xSKEtho3c4okKh} \\

(You may want to download it for convenience.)

Focusing on just the results of \texttt{seed\_1}, we can look at the posterior
distribution of $R_i$ (Figure~\ref{fig:R-post}) under different scales (0, 1, 10) and
different $K$ (5, 6, 7, 8, 15). Recall that $K^\true=7$ and $R^\true = (6, 5)$.

\begin{figure}[h]
  \begin{center}
    \includegraphics[scale=.5]{img/test-sim-6-5-1/metrics/seed_1/R.pdf}
  \end{center}
  \label{fig:R-post}
  \caption{Posterior distribution of $R_i$ for various $K$ and $\phi$.}
\end{figure}

Notice that models with higher $\phi$ (penalty term for repFAM), tend to have
smaller $R_i$. In addition, when $\phi=0$, redundant features are discovered.
For example, for $K_\text{MCMC} = 8$, we would expect to recover the simulation
truth for $Z$ as $K^\true = 7 \le K_\text{MCMC}$. The posterior mean of the
recovered $Z$ is as shown in Figure~\ref{fig:kmcmc8-scale0-zmean}. Notice that
features 2 and 7 are the same.  However, they are both selected. This can be seen
in Figure~\ref{fig:kmcmc8-scale0-w}, which shows the posterior distribution of $W$.
The weights for features 2 and 7 are both non-zero, leading to $R_1$ be between
5 and 6, but is misleading because one of the features is repeated.

\begin{figure}[h]
  \begin{center}
    \includegraphics[scale=.5]{img/test-sim-6-5-1/seed_1/scale_0/Kmcmc_8/img/Zmean.pdf}
  \end{center}
  \caption{Posterior mean of $Z$ for $\phi=0$, $K_\text{MCMC}=8$, and
  simulation data 1.}
  \label{fig:kmcmc8-scale0-zmean}
\end{figure}

\begin{figure}[h]
  \begin{center}
    \includegraphics[scale=.5]{img/test-sim-6-5-1/seed_1/scale_0/Kmcmc_8/img/W.pdf}
  \end{center}
  \caption{Posterior distribution of $W$ for $\phi=0$, $K_\text{MCMC}=8$, and
  simulation data 1.}
  \label{fig:kmcmc8-scale0-w}
\end{figure}

%%% 
In contrast, for $K_\text{MCMC} = 8$ and $\phi=1$, The posterior mean of the
recovered $Z$ is as shown in Figure~\ref{fig:kmcmc8-scale1-zmean}. Notice that
no features are duplicated.  And the simulation truth for $W$ is recovered.
This is true also for ($K_\text{MCMC}=15$, $\phi=1$) and ($K_\text{MCMC}=15$,
$\phi=10$).  However, the simulation truth was not recovered for
($K_\text{MCMC}=7$, $\phi=1$), ($K_\text{MCMC}=7$, $\phi=10$), and
($K_\text{MCMC}=8$, $\phi=10$).

\begin{figure}[h]
  \begin{center}
    \includegraphics[scale=.5]{img/test-sim-6-5-1/seed_1/scale_1/Kmcmc_8/img/Zmean.pdf}
  \end{center}
  \caption{Posterior mean of $Z$ for $\phi=1$, $K_\text{MCMC}=8$, and
  simulation data 1.}
  \label{fig:kmcmc8-scale1-zmean}
\end{figure}

\begin{figure}[h]
  \begin{center}
    \includegraphics[scale=.5]{img/test-sim-6-5-1/seed_1/scale_1/Kmcmc_8/img/W.pdf}
  \end{center}
  \caption{Posterior distribution of $W$ for $\phi=1$, $K_\text{MCMC}=8$, and
  simulation data 1.}
  \label{fig:kmcmc8-scale1-w}
\end{figure}


\end{document}
