\documentclass[10pt]{article} % 12-point font

\usepackage[margin=1.5cm]{geometry} % set page to 1-inch margins
\usepackage{amsmath} % for math
\usepackage{amssymb} % like \Rightarrow
\setlength\parindent{0pt} % Suppresses the indentation of new paragraphs.

% Big display
\newcommand{\ds}{ \displaystyle }
% Parenthesis
\newcommand{\norm}[1]{\left\lVert#1\right\rVert}
\newcommand{\p}[1]{\left(#1\right)}
\newcommand{\bk}[1]{\left[#1\right]}
\newcommand{\bc}[1]{ \left\{#1\right\} }
\newcommand{\abs}[1]{ \left|#1\right| }
% Derivatives
\newcommand{\df}[2]{ \frac{d#1}{d#2} }
\newcommand{\ddf}[2]{ \frac{d^2#1}{d{#2}^2} }
\newcommand{\pd}[2]{ \frac{\partial#1}{\partial#2} }
\newcommand{\pdd}[2]{\frac{\partial^2#1}{\partial{#2}^2} }
% Distributions
\newcommand{\Normal}{ \text{Normal} }
\newcommand{\Beta}{ \text{Beta} }
\newcommand{\Gam}{ \text{Gamma} }
\newcommand{\InvGamma}{ \text{Inv-Gamma} }
\newcommand{\Uniform}{ \text{Uniform} }
\def\Dir{\text{Dirichlet}}
\def\TN{\text{TN}}
% Statistics
\newcommand{\E}{\text{E}}
\newcommand{\iid}{\overset{iid}{\sim}}
\newcommand{\ind}{\overset{ind}{\sim}}

% Graphics
\usepackage{graphicx}  % for figures
\usepackage{float} % Put figure exactly where I want [H]

% Uncomment if using bibliography
% Bibliography
% \usepackage{natbib}
% \bibliographystyle{plainnat}

% Adds settings for hyperlinks. (Mainly for table of contents.)
\usepackage{hyperref}
\hypersetup{
  pdfborder={0 0 0} % removes red box from links
}


% Macros for this project
\def\true{\text{TRUE}}
\usepackage[dvipsnames,usenames]{color}
\newcommand{\bbh}{\color{blue}\textbf}  % blue bold highlight
\newcommand{\bch}{\color{blue}\it}  % blue italics
\newcommand{\ech}{\color{black}\rm}
\newcommand{\imgtemplate}[3]{
  \begin{tabular}{cccccc}
    {\tiny (a) Data1, MCMC1, $\phi=0$} &
    {\tiny (b) Data1, MCMC1, $\phi=1$}&
    {\tiny (c) Data1, MCMC1, $\phi=10$} &
    {\tiny (d) Data1, MCMC2, $\phi=0$} &
    {\tiny (e) Data1, MCMC2, $\phi=1$} &
    {\tiny (f) Data1, MCMC2, $\phi=10$} \\
    %
    \includegraphics[scale=#2]{\imgdir/dataseed1-mcmcseed1-phi0.0#3/img/#1} &
    \includegraphics[scale=#2]{\imgdir/dataseed1-mcmcseed1-phi1.0#3/img/#1} &
    \includegraphics[scale=#2]{\imgdir/dataseed1-mcmcseed1-phi10.0#3/img/#1} &
    %
    \includegraphics[scale=#2]{\imgdir/dataseed1-mcmcseed2-phi0.0#3/img/#1} &
    \includegraphics[scale=#2]{\imgdir/dataseed1-mcmcseed2-phi1.0#3/img/#1} &
    \includegraphics[scale=#2]{\imgdir/dataseed1-mcmcseed2-phi10.0#3/img/#1} \\
    %
    {\tiny (g) Data1, MCMC3, $\phi=0$} &
    {\tiny (h) Data1, MCMC3, $\phi=1$} &
    {\tiny (i) Data1, MCMC3, $\phi=10$} &
    {\tiny (j) Data2, MCMC1, $\phi=0$} &
    {\tiny (k) Data2, MCMC1, $\phi=1$} &
    {\tiny (l) Data2, MCMC1, $\phi=10$} \\
    %
    \includegraphics[scale=#2]{\imgdir/dataseed1-mcmcseed3-phi0.0#3/img/#1} &
    \includegraphics[scale=#2]{\imgdir/dataseed1-mcmcseed3-phi1.0#3/img/#1} &
    \includegraphics[scale=#2]{\imgdir/dataseed1-mcmcseed3-phi10.0#3/img/#1} &
    %
    \includegraphics[scale=#2]{\imgdir/dataseed2-mcmcseed1-phi0.0#3/img/#1} &
    \includegraphics[scale=#2]{\imgdir/dataseed2-mcmcseed1-phi1.0#3/img/#1} &
    \includegraphics[scale=#2]{\imgdir/dataseed2-mcmcseed1-phi10.0#3/img/#1} \\
    %
    {\tiny (m) Data2, MCMC2, $\phi=0$} &
    {\tiny (n) Data2, MCMC2, $\phi=1$} &
    {\tiny (o) Data2, MCMC2, $\phi=10$} &
    {\tiny (p) Data2, MCMC3, $\phi=0$} &
    {\tiny (q) Data2, MCMC3, $\phi=1$} &
    {\tiny (r) Data2, MCMC3, $\phi=10$} \\
    %
    \includegraphics[scale=#2]{\imgdir/dataseed2-mcmcseed2-phi0.0#3/img/#1} &
    \includegraphics[scale=#2]{\imgdir/dataseed2-mcmcseed2-phi1.0#3/img/#1} &
    \includegraphics[scale=#2]{\imgdir/dataseed2-mcmcseed2-phi10.0#3/img/#1} &
    %
    \includegraphics[scale=#2]{\imgdir/dataseed2-mcmcseed3-phi0.0#3/img/#1} &
    \includegraphics[scale=#2]{\imgdir/dataseed2-mcmcseed3-phi1.0#3/img/#1} &
    \includegraphics[scale=#2]{\imgdir/dataseed2-mcmcseed3-phi10.0#3/img/#1} \\
    %
    {\tiny (s) Data3, MCMC1, $\phi=0$} &
    {\tiny (t) Data3, MCMC1, $\phi=1$} &
    {\tiny (u) Data3, MCMC1, $\phi=10$} &
    {\tiny (v) Data3, MCMC2, $\phi=0$} &
    {\tiny (w) Data3, MCMC2, $\phi=1$} &
    {\tiny (x) Data3, MCMC2, $\phi=10$} \\
    %
    \includegraphics[scale=#2]{\imgdir/dataseed3-mcmcseed1-phi0.0#3/img/#1} &
    \includegraphics[scale=#2]{\imgdir/dataseed3-mcmcseed1-phi1.0#3/img/#1} &
    \includegraphics[scale=#2]{\imgdir/dataseed3-mcmcseed1-phi10.0#3/img/#1} &
    %
    \includegraphics[scale=#2]{\imgdir/dataseed3-mcmcseed2-phi0.0#3/img/#1} &
    \includegraphics[scale=#2]{\imgdir/dataseed3-mcmcseed2-phi1.0#3/img/#1} &
    \includegraphics[scale=#2]{\imgdir/dataseed3-mcmcseed2-phi10.0#3/img/#1} \\
    %
    {\tiny (y) Data3, MCMC3, $\phi=0$} &
    {\tiny (z) Data3, MCMC3, $\phi=1$} &
    {\tiny ($\alpha$) Data3, MCMC3, $\phi=10$} &
    & & \\
    %
    \includegraphics[scale=#2]{\imgdir/dataseed3-mcmcseed3-phi0.0#3/img/#1} &
    \includegraphics[scale=#2]{\imgdir/dataseed3-mcmcseed3-phi1.0#3/img/#1} &
    \includegraphics[scale=#2]{\imgdir/dataseed3-mcmcseed3-phi10.0#3/img/#1} \\
    & & \\
  \end{tabular}
}

\def\imgdir{../../results/test-sim-6-7-4/maxtemp2-ntempts20-degree2-N500}

% Title Settings
\title{Simulation Study 6.7.4}
\author{Arthur Lui}
\date{\today} % \date{} to set date to empty

% MAIN %
\begin{document}

\maketitle

% \tableofcontents \newpage % Comment to remove table of contents

% \section{Objective}\label{sec:objective}
% TODO

\section{Data Generation}\label{sec:data-generation}
Figure~\ref{fig:Z-true} and Table~\ref{tab:W-true} respectively show the true
$Z$ and $W$ used to generate the simulated data in this study. Note the following:
\begin{itemize}
  \item Features (columns) 1, 2, and 3 are similar (maximum difference of 2).
    In Sample 1, the first two features are rare / absent in one sample, and
    the third feature is abundant in both samples. Previously, we wanted to
    know if the first two features are prone to being absorbed into feature 3. We
    found that using the prior $w^\star_{i,k} \sim \Gam(10, 1)$, and using priors
    that encouraged $r_{i,k}$ to be sparse, Features 1 and 2 tend to merge
    (into Feature 2, which is slightly larger), and Feature 3 would be in its
    own cluster or split into two smaller identical features. We believe this is 
    due to prior information in $w^\star$ encouraging features to be equally weighted.
    In this study, we continue to monitor these kinds of behaviour.
  \item Features 4 and 5 are similar (difference of 1). They are rare / absent
    in one sample and abundant in another. Previously, we were also curious if
    these two features will be merged. Although, we didn't expect them to. It was
    conceivable that they will only be ``selected'' in one sample, but not in
    another. We found that they were always recovered, in part, we believe, because
    they had a large presence in one sample. We continue to monitor their behaviour
    here.
  \item Feature 6 is abundant in Sample 1 but not in Sample 2. We expect it to be
    recovered in $Z$ but it is conceivable that it will not be ``selected'' in sample
    2. It is distinct from other features.
  \item Feature 7 is rare in Sample 1, and absent in Sample 2. Nevertheless, we
    expect it to recovered in $Z$ because it is different from the other
    features.
\end{itemize}

\begin{figure}[H]
  \begin{center}  % 6 x 5
    \includegraphics[scale=.4]{\imgdir/img/temper_1/Z_true.pdf}
  \end{center}
  \caption{$Z^\true$}
  \label{fig:Z-true}
\end{figure}

% latex table generated in R 3.4.4 by xtable 1.8-3 package
% Wed Feb 12 14:53:27 2020
\begin{table}[ht]
  \centering
  \begin{tabular}{rrrrrrrr}
    \hline
    & $k=1$ & $k=2$ & $k=3$ & $k=4$ & $k=5$ & $k=6$ & $k=7$ \\
    \hline
    Sample 1 & 0.04 & 0.05 & 0.39 & 0.00 & 0.06 & 0.43 & 0.03 \\
    Sample 2 & 0.00 & 0.05 & 0.54 & 0.20 & 0.14 & 0.07 & 0.00 \\
    \hline
  \end{tabular}
  \caption{$W^\true$}
  \label{tab:W-true}
\end{table}

\subsection{More on data generation / MCMC setup}
\begin{itemize}
  \item $(\mu_{0,j}^\star)^\true=-2 + \epsilon_j$, $\epsilon_j \sim \Uniform(-0.5, 0.5)$
  \item $(\mu_{1,j}^\star)^\true=2 + \epsilon_j$, $\epsilon_j \sim \Uniform(-0.5, 0.5)$
  \item ($\sigma^2_i)^\true=0.5$, $N=(500, 500)$
  \item 10000 burn-in, followed by 4000 iterations, and every other sample was
    kept (2000 samples total).
  \item \textbf{Some Priors:}
  \begin{itemize}
    \item $w^\star_{i,k} \sim \Gam(\text{shape}=10, \text{rate}=1)$
    \item $\eta_{z, i,j,\ell} \sim \Dir_{L_z}(5)$
    \item $\alpha \sim \Gam(0.1, 0.1)$
    \item $\delta_0 \sim \TN^-(1.0, 0.1)$
    \item $\delta_1 \sim \TN^+(1.0, 0.1)$
    \item $\sigma^2_i \sim \InvGamma^+(1.0, 0.1)$
  \end{itemize}
\item \textbf{Temperatures}: 
    \begin{itemize}
      \item 
        1, 1, 1, 1.006, 1.017,
        1.032, 1.052, 1.078,
        1.108, 1.144, 1.187,
        1.236, 1.293, 1.359,
        1.434, 1.519, 1.617,
        1.728, 1.855, 2
      \item Note that the first three temperatures are 1.
      \item The last 17 temperatures are generated from $2^{(i^2 - 1) / (n^2 -
        1)}$, where $n=17$ is the number of temperatures, and
        $i\in\bc{1,\cdots,n}$.
    \end{itemize}
\end{itemize}

% \newpage
\section{Swap probabilities (post-burn)}
\begin{figure}[H]
  \centering
  \includegraphics[scale=.7]{\imgdir/img/swapprops.pdf}
  \caption{This heatmap shows the proportion of instances where two
  temperature chains were swapped when they were proposed to be swapped. The
  ticks on the axes indicate the temperatures. Swapping is occurring between
  most adjacent pairs, and sometimes non-adjacent pairs. But for temperatures
  (6, 7), (11, 12), (13, 14), and (18, 19), swapping does not occur. The reason
  for this is that the parameters are in different modes (see $\mu, \sigma^2,
  Z$). Making the temperature grid finer is necessary to achieve better mixing
  (propagation of states from hot to cold chains). Note that $\tau=1$ occurs
  three times. And the probability of swapping is always 1, by construction.}
\end{figure}

\newpage
\section{Log likelihood (post-burn)}
\begin{figure}[H]
  \begin{center}  % 6 x 5
    \begin{tabular}{cccc}
      \includegraphics[scale=.25]{\imgdir/img/loglike/loglike_postburn_1.pdf} &
      \includegraphics[scale=.25]{\imgdir/img/loglike/loglike_postburn_2.pdf} &
      \includegraphics[scale=.25]{\imgdir/img/loglike/loglike_postburn_3.pdf} &
      \includegraphics[scale=.25]{\imgdir/img/loglike/loglike_postburn_4.pdf} \\
      \includegraphics[scale=.25]{\imgdir/img/loglike/loglike_postburn_5.pdf} &
      \includegraphics[scale=.25]{\imgdir/img/loglike/loglike_postburn_6.pdf} &
      \includegraphics[scale=.25]{\imgdir/img/loglike/loglike_postburn_7.pdf} &
      \includegraphics[scale=.25]{\imgdir/img/loglike/loglike_postburn_8.pdf} \\
      \includegraphics[scale=.25]{\imgdir/img/loglike/loglike_postburn_9.pdf} &
      \includegraphics[scale=.25]{\imgdir/img/loglike/loglike_postburn_10.pdf} &
      \includegraphics[scale=.25]{\imgdir/img/loglike/loglike_postburn_11.pdf} &
      \includegraphics[scale=.25]{\imgdir/img/loglike/loglike_postburn_12.pdf} \\
      \includegraphics[scale=.25]{\imgdir/img/loglike/loglike_postburn_13.pdf} &
      \includegraphics[scale=.25]{\imgdir/img/loglike/loglike_postburn_14.pdf} &
      \includegraphics[scale=.25]{\imgdir/img/loglike/loglike_postburn_15.pdf} &
      \includegraphics[scale=.25]{\imgdir/img/loglike/loglike_postburn_16.pdf} \\
      \includegraphics[scale=.25]{\imgdir/img/loglike/loglike_postburn_17.pdf} &
      \includegraphics[scale=.25]{\imgdir/img/loglike/loglike_postburn_18.pdf} &
      \includegraphics[scale=.25]{\imgdir/img/loglike/loglike_postburn_19.pdf} &
      \includegraphics[scale=.25]{\imgdir/img/loglike/loglike_postburn_20.pdf} \\
    \end{tabular}
  \end{center}
  \caption{Post burn-in log likelihoods for each run. Burn in used was 10000.
    These plots show the log-likelihood evaluations of the 4000 subsequent
    iterations for each temperature, for $\phi=1$.}
  \label{fig:ll}
\end{figure}

\newpage
\section{Posterior estimate of $Z_1$}
\begin{figure}[H]
  \begin{center}  % 6 x 5
    \imgtemplate{Z1.pdf}{.2}
  \end{center}
  \caption{Note that subpopulation 3 for $Z_1$ and subpopulation 1 for $Z_2$
  each contain two subpopulations. Also note that the columns in $Z$ are different.}
  \label{fig:yzest}
\end{figure}

\newpage
\section{Posterior estimate of $Z_2$}
\begin{figure}[H]
  \begin{center}  % 6 x 5
    \imgtemplate{Z2.pdf}{.2}
  \end{center}
  \caption{Note that subpopulation 3 for $Z_1$ and subpopulation 1 for $Z_2$
  each contain two subpopulations. Also note that the columns in $Z$ are different.}
  \label{fig:yzest}
\end{figure}

\newpage
\section{Posterior estimate of $y_1$}
\begin{figure}[H]
  \begin{center}  % 6 x 5
    \imgtemplate{y1.pdf}{.2}
  \end{center}
  \caption{Note that subpopulation 3 for $Z_1$ and subpopulation 1 for $Z_2$
  each contain two subpopulations. Also note that the columns in $Z$ are different.}
  \label{fig:yzest}
\end{figure}

\newpage
\section{Posterior estimate of $y_2$}
\begin{figure}[H]
  \begin{center}  % 6 x 5
    \imgtemplate{y2.pdf}{.2}
  \end{center}
  \caption{Note that subpopulation 3 for $Z_1$ and subpopulation 1 for $Z_2$
  each contain two subpopulations. Also note that the columns in $Z$ are different.}
  \label{fig:yzest}
\end{figure}


\newpage
\section{Trace plots of $\mu^\star$}
\begin{figure}[H]
  \begin{center}  % 6 x 5
    \imgtemplate{mus_trace.pdf}{.2}
  \end{center}
  \label{fig:mustrace}
  \caption{}
\end{figure}

\newpage
\section{Posterior Distribution of $\mu^\star$}
\begin{figure}[H]
  \begin{center}  % 6 x 5
    \imgtemplate{mus.pdf}{.2}
  \end{center}
  \label{fig:mus}
  \caption{}
\end{figure}

\newpage
\section{Trace plots of $\sigma^2$}
\begin{figure}[H]
  \begin{center}  % 6 x 5
    \imgtemplate{sig2_trace.pdf}{.2}
  \end{center}
  \label{fig:sig2trace}
  \caption{}
\end{figure}

\newpage
\section{Posterior Distribution of $\sigma^2$}
\begin{figure}[H]
  \begin{center}  % 6 x 5
    \imgtemplate{sig2.pdf}{.2}
  \end{center}
  \label{fig:sig2}
  \caption{}
\end{figure}



\newpage
\section{Posterior predictive density for sample 1, marker 1}
\begin{figure}[H]
  \begin{center}  % 6 x 5
    \begin{tabular}{ccc}
      {(a) sample 1, marker 1} &
      {(b) sample 1, marker 2} &
      {(c) sample 1, marker 3} \\
      \includegraphics[scale=.3]{\imgdir/img/dden/dden_i1_j1.pdf} &
      \includegraphics[scale=.3]{\imgdir/img/dden/dden_i1_j2.pdf} &
      \includegraphics[scale=.3]{\imgdir/img/dden/dden_i1_j3.pdf} \\
      {(d) sample 2, marker 1} &
      {(e) sample 2, marker 2} &
      {(f) sample 2, marker 3} \\
      \includegraphics[scale=.3]{\imgdir/img/dden/dden_i2_j1.pdf} &
      \includegraphics[scale=.3]{\imgdir/img/dden/dden_i2_j2.pdf} &
      \includegraphics[scale=.3]{\imgdir/img/dden/dden_i2_j3.pdf} \\
    \end{tabular}
  \end{center}
  \label{fig:dd11}
  \caption{}
\end{figure}
 
% \newpage
\section{Missing Mechanism}
\begin{figure}[H]
  \begin{center}  % 6 x 5
    \begin{tabular}{cc}
      \includegraphics[scale=.5]{\imgdir/img/missmech_1.pdf} &
      \includegraphics[scale=.5]{\imgdir/img/missmech_2.pdf} \\
    \end{tabular}
  \end{center}
\label{fig:sigtrace}
\caption{}
\end{figure}


% Uncomment if using bibliography:
% \bibliography{bib}
\end{document}
