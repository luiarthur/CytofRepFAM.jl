\documentclass[10pt]{article} % 12-point font

\usepackage[margin=1cm]{geometry} % set page to 1-inch margins
\usepackage{amsmath} % for math
\usepackage{amssymb} % like \Rightarrow
\setlength\parindent{0pt} % Suppresses the indentation of new paragraphs.
\pagenumbering{gobble} % suppress page numbers

% Big display
\newcommand{\ds}{ \displaystyle }
% Parenthesis
\newcommand{\norm}[1]{\left\lVert#1\right\rVert}
\newcommand{\p}[1]{\left(#1\right)}
\newcommand{\bk}[1]{\left[#1\right]}
\newcommand{\bc}[1]{ \left\{#1\right\} }
\newcommand{\abs}[1]{ \left|#1\right| }
% Derivatives
\newcommand{\df}[2]{ \frac{d#1}{d#2} }
\newcommand{\ddf}[2]{ \frac{d^2#1}{d{#2}^2} }
\newcommand{\pd}[2]{ \frac{\partial#1}{\partial#2} }
\newcommand{\pdd}[2]{\frac{\partial^2#1}{\partial{#2}^2} }
% Distributions
\newcommand{\Normal}{ \text{Normal} }
\newcommand{\Beta}{ \text{Beta} }
\newcommand{\Gam}{ \text{Gamma} }
\newcommand{\InvGamma}{ \text{Inv-Gamma} }
\newcommand{\Uniform}{ \text{Uniform} }
\def\Dir{\text{Dirichlet}}
\def\TN{\text{TN}}
% Statistics
\newcommand{\E}{\text{E}}
\newcommand{\iid}{\overset{iid}{\sim}}
\newcommand{\ind}{\overset{ind}{\sim}}

% Graphics
\usepackage{graphicx}  % for figures
\usepackage{float} % Put figure exactly where I want [H]

% Uncomment if using bibliography
% Bibliography
% \usepackage{natbib}
% \bibliographystyle{plainnat}

% Adds settings for hyperlinks. (Mainly for table of contents.)
\usepackage{hyperref}
\hypersetup{
  pdfborder={0 0 0} % removes red box from links
}


% Macros for this project
\def\true{\text{TRUE}}
\usepackage[dvipsnames,usenames]{color}
\newcommand{\bbh}{\color{blue}\textbf}  % blue bold highlight
\newcommand{\bch}{\color{blue}\it}  % blue italics
\newcommand{\ech}{\color{black}\rm}
\def\imgdir{../../results/test-sim-6-7-9}
\newcommand{\imgtemplate}[3]{
  \begin{tabular}{cccccc}
    {\tiny (a) Data1, MCMC1, $\phi=0$} &
    {\tiny (b) Data1, MCMC1, $\phi=1$}&
    {\tiny (c) Data1, MCMC1, $\phi=10$} &
    {\tiny (d) Data1, MCMC2, $\phi=0$} &
    {\tiny (e) Data1, MCMC2, $\phi=1$} &
    {\tiny (f) Data1, MCMC2, $\phi=10$} \\
    %
    \includegraphics[scale=#2]{\imgdir/dataseed1-mcmcseed1-phi0.0#3/img/#1} &
    \includegraphics[scale=#2]{\imgdir/dataseed1-mcmcseed1-phi1.0#3/img/#1} &
    \includegraphics[scale=#2]{\imgdir/dataseed1-mcmcseed1-phi10.0#3/img/#1} &
    %
    \includegraphics[scale=#2]{\imgdir/dataseed1-mcmcseed2-phi0.0#3/img/#1} &
    \includegraphics[scale=#2]{\imgdir/dataseed1-mcmcseed2-phi1.0#3/img/#1} &
    \includegraphics[scale=#2]{\imgdir/dataseed1-mcmcseed2-phi10.0#3/img/#1} \\
    %
    {\tiny (g) Data1, MCMC3, $\phi=0$} &
    {\tiny (h) Data1, MCMC3, $\phi=1$} &
    {\tiny (i) Data1, MCMC3, $\phi=10$} &
    {\tiny (j) Data2, MCMC1, $\phi=0$} &
    {\tiny (k) Data2, MCMC1, $\phi=1$} &
    {\tiny (l) Data2, MCMC1, $\phi=10$} \\
    %
    \includegraphics[scale=#2]{\imgdir/dataseed1-mcmcseed3-phi0.0#3/img/#1} &
    \includegraphics[scale=#2]{\imgdir/dataseed1-mcmcseed3-phi1.0#3/img/#1} &
    \includegraphics[scale=#2]{\imgdir/dataseed1-mcmcseed3-phi10.0#3/img/#1} &
    %
    \includegraphics[scale=#2]{\imgdir/dataseed2-mcmcseed1-phi0.0#3/img/#1} &
    \includegraphics[scale=#2]{\imgdir/dataseed2-mcmcseed1-phi1.0#3/img/#1} &
    \includegraphics[scale=#2]{\imgdir/dataseed2-mcmcseed1-phi10.0#3/img/#1} \\
    %
    {\tiny (m) Data2, MCMC2, $\phi=0$} &
    {\tiny (n) Data2, MCMC2, $\phi=1$} &
    {\tiny (o) Data2, MCMC2, $\phi=10$} &
    {\tiny (p) Data2, MCMC3, $\phi=0$} &
    {\tiny (q) Data2, MCMC3, $\phi=1$} &
    {\tiny (r) Data2, MCMC3, $\phi=10$} \\
    %
    \includegraphics[scale=#2]{\imgdir/dataseed2-mcmcseed2-phi0.0#3/img/#1} &
    \includegraphics[scale=#2]{\imgdir/dataseed2-mcmcseed2-phi1.0#3/img/#1} &
    \includegraphics[scale=#2]{\imgdir/dataseed2-mcmcseed2-phi10.0#3/img/#1} &
    %
    \includegraphics[scale=#2]{\imgdir/dataseed2-mcmcseed3-phi0.0#3/img/#1} &
    \includegraphics[scale=#2]{\imgdir/dataseed2-mcmcseed3-phi1.0#3/img/#1} &
    \includegraphics[scale=#2]{\imgdir/dataseed2-mcmcseed3-phi10.0#3/img/#1} \\
    %
    {\tiny (s) Data3, MCMC1, $\phi=0$} &
    {\tiny (t) Data3, MCMC1, $\phi=1$} &
    {\tiny (u) Data3, MCMC1, $\phi=10$} &
    {\tiny (v) Data3, MCMC2, $\phi=0$} &
    {\tiny (w) Data3, MCMC2, $\phi=1$} &
    {\tiny (x) Data3, MCMC2, $\phi=10$} \\
    %
    \includegraphics[scale=#2]{\imgdir/dataseed3-mcmcseed1-phi0.0#3/img/#1} &
    \includegraphics[scale=#2]{\imgdir/dataseed3-mcmcseed1-phi1.0#3/img/#1} &
    \includegraphics[scale=#2]{\imgdir/dataseed3-mcmcseed1-phi10.0#3/img/#1} &
    %
    \includegraphics[scale=#2]{\imgdir/dataseed3-mcmcseed2-phi0.0#3/img/#1} &
    \includegraphics[scale=#2]{\imgdir/dataseed3-mcmcseed2-phi1.0#3/img/#1} &
    \includegraphics[scale=#2]{\imgdir/dataseed3-mcmcseed2-phi10.0#3/img/#1} \\
    %
    {\tiny (y) Data3, MCMC3, $\phi=0$} &
    {\tiny (z) Data3, MCMC3, $\phi=1$} &
    {\tiny ($\alpha$) Data3, MCMC3, $\phi=10$} &
    & & \\
    %
    \includegraphics[scale=#2]{\imgdir/dataseed3-mcmcseed3-phi0.0#3/img/#1} &
    \includegraphics[scale=#2]{\imgdir/dataseed3-mcmcseed3-phi1.0#3/img/#1} &
    \includegraphics[scale=#2]{\imgdir/dataseed3-mcmcseed3-phi10.0#3/img/#1} \\
    & & \\
  \end{tabular}
}


% Title Settings
\title{Simulation Study 6.7.9}
\author{Arthur Lui}
\date{\today} % \date{} to set date to empty

% MAIN %
\begin{document}

\maketitle

% \tableofcontents \newpage % Comment to remove table of contents

% \section{Objective}\label{sec:objective}
% TODO

\section{Data Generation}\label{sec:data-generation}
Figure~\ref{fig:Z-true} and Table~\ref{tab:W-true} respectively show the true
$Z$ and $W$ used to generate the simulated data in this study. Note the following:
\begin{itemize}
  \item Features (columns) 1, 2, and 3 are similar (maximum difference of 2).
    In Sample 1, the first two features are rare / absent in one sample, and
    the third feature is abundant in both samples. Previously, we wanted to
    know if the first two features are prone to being absorbed into feature 3. We
    found that using the prior $w^\star_{i,k} \sim \Gam(10, 1)$, and using priors
    that encouraged $r_{i,k}$ to be sparse, Features 1 and 2 tend to merge
    (into Feature 2, which is slightly larger), and Feature 3 would be in its
    own cluster or split into two smaller identical features. We believe this is 
    due to prior information in $w^\star$ encouraging features to be equally weighted.
    In this study, we continue to monitor these kinds of behaviour.
  \item Features 4 and 5 are similar (difference of 1). They are rare / absent
    in one sample and abundant in another. Previously, we were also curious if
    these two features will be merged. Although, we didn't expect them to. It was
    conceivable that they will only be ``selected'' in one sample, but not in
    another. We found that they were always recovered, in part, we believe, because
    they had a large presence in one sample. We continue to monitor their behaviour
    here.
  \item Feature 6 is abundant in Sample 1 but not in Sample 2. We expect it to be
    recovered in $Z$ but it is conceivable that it will not be ``selected'' in sample
    2. It is distinct from other features.
  \item Feature 7 is rare in Sample 1, and absent in Sample 2. Nevertheless, we
    expect it to recovered in $Z$ because it is different from the other
    features.
\end{itemize}

\begin{figure}[H]
  \begin{center}  % 6 x 5
    \includegraphics[scale=.6]{\imgdir/dataseed1-mcmcseed1-phi0.0/img/Z_true.pdf}
  \end{center}
  \caption{$Z^\true$}
  \label{fig:Z-true}
\end{figure}

% latex table generated in R 3.4.4 by xtable 1.8-3 package
% Wed Feb 12 14:53:27 2020
\begin{table}[ht]
  \centering
  \begin{tabular}{rrrrrrrr}
    \hline
    & $k=1$ & $k=2$ & $k=3$ & $k=4$ & $k=5$ & $k=6$ & $k=7$ \\
    \hline
    Sample 1 & 0.04 & 0.05 & 0.39 & 0.00 & 0.06 & 0.43 & 0.03 \\
    Sample 2 & 0.00 & 0.05 & 0.54 & 0.20 & 0.14 & 0.07 & 0.00 \\
    \hline
  \end{tabular}
  \caption{$W^\true$}
  \label{tab:W-true}
\end{table}

\subsection{More on data generation / MCMC setup}
\begin{itemize}
  \item $(\mu_{0,j}^\star)^\true=-2 + \epsilon_j$, $\epsilon_j \sim \Uniform(-0.5, 0.5)$
  \item $(\mu_{1,j}^\star)^\true=2 + \epsilon_j$, $\epsilon_j \sim \Uniform(-0.5, 0.5)$
  \item ($\sigma^2_i)^\true=0.5$, $N=(2000, 2000)$
  \item 500 burn-in, followed by 2000 iterations, and every other sample was
    kept (1000 samples total).
  \item Computation time for each chains was was approximately 14 hours.
  \item \textbf{Some Priors / Setup:}
  \begin{itemize}
    \item $w^\star_{i,k} \sim \Gam(\text{shape}=1/2, \text{rate}=1/2)$
    \item $p_c \sim \text{Beta}(1, 1)$
    \item $\eta_{z, i,j,\ell} \sim \Dir_{L_z}(5)$
    \item $\alpha \sim \Gam(0.1, 0.1)$
    \item $\delta_0 \sim \TN^-(1.0, 0.1)$
    \item $\delta_1 \sim \TN^+(1.0, 0.1)$
    \item $\sigma^2_i \sim \InvGamma^+(1.0, 0.1)$
    \item $\phi\in\bc{0, 1, 10}$
    \item Three datasets were generated with same settings, but different
      random seed.
    \item Three chains (different random seeds) were run for each $\phi$ and data.

    \item 1\% of data in each sample are used to sample from trained prior, and
      $M=4$ was used.
  \end{itemize}
\end{itemize}

\newpage
\section{Log likelihood (post-burn)}
\begin{figure}[H]
  \begin{center}  % 6 x 5
    \imgtemplate{loglike_postburn.pdf}{0.15}
  \end{center}
  \caption{Post burn-in log likelihoods for each run. Burn in used was 500.
    The plot shows the log-likelihood evaluations of the 2000 subsequent
    iterations.}
  \label{fig:ll}
\end{figure}

\newpage
\section{Posterior mean of $Z$}
\begin{figure}[H]
  \begin{center}  % 6 x 5
    \imgtemplate{Zmean.pdf}{0.15}
  \end{center}
  % TODO
  \caption{}
  \label{fig:zmean}
\end{figure}
 
\newpage
\section{Posterior estimate of $Z_1$}
\begin{figure}[H]
  \begin{center}  % 6 x 5
    \imgtemplate{Z1.pdf}{0.15}
  \end{center}
  % TODO
  \caption{}
  \label{fig:z1est}
\end{figure}

\newpage
\section{Posterior estimate of $Z_2$}
\begin{figure}[H]
  \begin{center}  % 6 x 5
    \imgtemplate{Z2.pdf}{0.15}
  \end{center}
  % TODO
  \caption{}
  \label{fig:z2est}
\end{figure}

\newpage
\section{Posterior estimate of $y_1$}
\begin{figure}[H]
  \begin{center}  % 6 x 5
    \imgtemplate{y1.pdf}{0.15}
  \end{center}
  % TODO
  \caption{}
  \label{fig:y1est}
\end{figure}

\newpage
\section{Posterior estimate of $y_2$}
\begin{figure}[H]
  \begin{center}  % 6 x 5
    \imgtemplate{y2.pdf}{0.15}
  \end{center}
  % TODO
  \caption{}
  \label{fig:y2est}
\end{figure}

\newpage
\section{Trace plots for $\mu^\star$}
\begin{figure}[H]
  \begin{center}  % 6 x 5
    \imgtemplate{mus_trace.pdf}{0.15}
  \end{center}
  % TODO
  \caption{}
  \label{fig:mus-trace}
\end{figure}

\newpage
\section{Posterior distributions for $\mu^\star$}
\begin{figure}[H]
  \begin{center}  % 6 x 5
    \imgtemplate{mus.pdf}{0.15}
  \end{center}
  % TODO
  \caption{}
  \label{fig:mus}
\end{figure}

\newpage
\section{Trace plots for $\sigma^2$}
\begin{figure}[H]
  \begin{center}  % 6 x 5
    \imgtemplate{sig2_trace.pdf}{0.15}
  \end{center}
  % TODO
  \caption{}
  \label{fig:sig2-trace}
\end{figure}

\newpage
\section{Posterior distributions for $\sigma^2$}
\begin{figure}[H]
  \begin{center}  % 6 x 5
    \imgtemplate{sig2.pdf}{0.15}
  \end{center}
  % TODO
  \caption{}
  \label{fig:sig2}
\end{figure}

\newpage
\section{Posterior distributions for sample 1 marker 1}
\begin{figure}[H]
  \begin{center}  % 6 x 5
    \imgtemplate{dden/dden_i1_j1.pdf}{0.15}
  \end{center}
  % TODO
  \caption{}
  \label{fig:ddi1j1}
\end{figure}

\newpage
\section{Posterior distributions for sample 1 marker 2}
\begin{figure}[H]
  \begin{center}  % 6 x 5
    \imgtemplate{dden/dden_i1_j2.pdf}{0.15}
  \end{center}
  % TODO
  \caption{}
  \label{fig:ddi1j2}
\end{figure}

\newpage
\section{Posterior distributions for sample 2 marker 1}
\begin{figure}[H]
  \begin{center}  % 6 x 5
    \imgtemplate{dden/dden_i2_j1.pdf}{0.15}
  \end{center}
  % TODO
  \caption{}
  \label{fig:ddi2j1}
\end{figure}

\newpage
\section{Posterior distributions for sample 2 marker 2}
\begin{figure}[H]
  \begin{center}  % 6 x 5
    \imgtemplate{dden/dden_i2_j2.pdf}{0.15}
  \end{center}
  % TODO
  \caption{}
  \label{fig:ddi2j2}
\end{figure}


\newpage
\section{Missing Mechanism}
\begin{figure}[H]
  \begin{center}  % 6 x 5
    \begin{tabular}{cc}
      \includegraphics[scale=.5]{\imgdir/dataseed1-mcmcseed1-phi0.0/img/missmech_1.pdf} &
      \includegraphics[scale=.5]{\imgdir/dataseed1-mcmcseed1-phi0.0/img/missmech_2.pdf} \\
    \end{tabular}
  \end{center}
\label{fig:missmech}
\end{figure}


% % Uncomment if using bibliography:
% % \bibliography{bib}
\end{document}
