\documentclass[10pt]{article} % 12-point font

\usepackage[margin=1.5cm]{geometry} % set page to 1-inch margins
\usepackage{amsmath} % for math
\usepackage{amssymb} % like \Rightarrow
\setlength\parindent{0pt} % Suppresses the indentation of new paragraphs.

% Big display
\newcommand{\ds}{ \displaystyle }
% Parenthesis
\newcommand{\norm}[1]{\left\lVert#1\right\rVert}
\newcommand{\p}[1]{\left(#1\right)}
\newcommand{\bk}[1]{\left[#1\right]}
\newcommand{\bc}[1]{ \left\{#1\right\} }
\newcommand{\abs}[1]{ \left|#1\right| }
% Derivatives
\newcommand{\df}[2]{ \frac{d#1}{d#2} }
\newcommand{\ddf}[2]{ \frac{d^2#1}{d{#2}^2} }
\newcommand{\pd}[2]{ \frac{\partial#1}{\partial#2} }
\newcommand{\pdd}[2]{\frac{\partial^2#1}{\partial{#2}^2} }
% Distributions
\newcommand{\Normal}{ \text{Normal} }
\newcommand{\Beta}{ \text{Beta} }
\newcommand{\G}{ \text{Gamma} }
\newcommand{\InvGamma}{ \text{Inv-Gamma} }
\newcommand{\Uniform}{ \text{Uniform} }
% Statistics
\newcommand{\E}{ \text{E} }
\newcommand{\iid}{\overset{iid}{\sim}}
\newcommand{\ind}{\overset{ind}{\sim}}

% Graphics
\usepackage{graphicx}  % for figures
\usepackage{float} % Put figure exactly where I want [H]

% Uncomment if using bibliography
% Bibliography
% \usepackage{natbib}
% \bibliographystyle{plainnat}

% Adds settings for hyperlinks. (Mainly for table of contents.)
\usepackage{hyperref}
\hypersetup{
  pdfborder={0 0 0} % removes red box from links
}


% Macros for this project
\def\true{\text{TRUE}}
\usepackage[dvipsnames,usenames]{color}
\newcommand{\bbh}{\color{blue}\textbf}  % blue bold highlight
\newcommand{\bch}{\color{blue}\it}  % blue italics
\newcommand{\ech}{\color{black}\rm}
% \usepackage{listings}
\def\imgdir{../../results/test-sim-6-5-2/}
\newcommand{\imgtemplate}[3]{
  \begin{tabular}{cccccc}
    {\tiny (a) Data1, MCMC1, $\phi=0$} &
    {\tiny (b) Data1, MCMC1, $\phi=1$}&
    {\tiny (c) Data1, MCMC1, $\phi=10$} &
    {\tiny (d) Data1, MCMC2, $\phi=0$} &
    {\tiny (e) Data1, MCMC2, $\phi=1$} &
    {\tiny (f) Data1, MCMC2, $\phi=10$} \\
    %
    \includegraphics[scale=#2]{\imgdir/dataseed1-mcmcseed1-phi0.0#3/img/#1} &
    \includegraphics[scale=#2]{\imgdir/dataseed1-mcmcseed1-phi1.0#3/img/#1} &
    \includegraphics[scale=#2]{\imgdir/dataseed1-mcmcseed1-phi10.0#3/img/#1} &
    %
    \includegraphics[scale=#2]{\imgdir/dataseed1-mcmcseed2-phi0.0#3/img/#1} &
    \includegraphics[scale=#2]{\imgdir/dataseed1-mcmcseed2-phi1.0#3/img/#1} &
    \includegraphics[scale=#2]{\imgdir/dataseed1-mcmcseed2-phi10.0#3/img/#1} \\
    %
    {\tiny (g) Data1, MCMC3, $\phi=0$} &
    {\tiny (h) Data1, MCMC3, $\phi=1$} &
    {\tiny (i) Data1, MCMC3, $\phi=10$} &
    {\tiny (j) Data2, MCMC1, $\phi=0$} &
    {\tiny (k) Data2, MCMC1, $\phi=1$} &
    {\tiny (l) Data2, MCMC1, $\phi=10$} \\
    %
    \includegraphics[scale=#2]{\imgdir/dataseed1-mcmcseed3-phi0.0#3/img/#1} &
    \includegraphics[scale=#2]{\imgdir/dataseed1-mcmcseed3-phi1.0#3/img/#1} &
    \includegraphics[scale=#2]{\imgdir/dataseed1-mcmcseed3-phi10.0#3/img/#1} &
    %
    \includegraphics[scale=#2]{\imgdir/dataseed2-mcmcseed1-phi0.0#3/img/#1} &
    \includegraphics[scale=#2]{\imgdir/dataseed2-mcmcseed1-phi1.0#3/img/#1} &
    \includegraphics[scale=#2]{\imgdir/dataseed2-mcmcseed1-phi10.0#3/img/#1} \\
    %
    {\tiny (m) Data2, MCMC2, $\phi=0$} &
    {\tiny (n) Data2, MCMC2, $\phi=1$} &
    {\tiny (o) Data2, MCMC2, $\phi=10$} &
    {\tiny (p) Data2, MCMC3, $\phi=0$} &
    {\tiny (q) Data2, MCMC3, $\phi=1$} &
    {\tiny (r) Data2, MCMC3, $\phi=10$} \\
    %
    \includegraphics[scale=#2]{\imgdir/dataseed2-mcmcseed2-phi0.0#3/img/#1} &
    \includegraphics[scale=#2]{\imgdir/dataseed2-mcmcseed2-phi1.0#3/img/#1} &
    \includegraphics[scale=#2]{\imgdir/dataseed2-mcmcseed2-phi10.0#3/img/#1} &
    %
    \includegraphics[scale=#2]{\imgdir/dataseed2-mcmcseed3-phi0.0#3/img/#1} &
    \includegraphics[scale=#2]{\imgdir/dataseed2-mcmcseed3-phi1.0#3/img/#1} &
    \includegraphics[scale=#2]{\imgdir/dataseed2-mcmcseed3-phi10.0#3/img/#1} \\
    %
    {\tiny (s) Data3, MCMC1, $\phi=0$} &
    {\tiny (t) Data3, MCMC1, $\phi=1$} &
    {\tiny (u) Data3, MCMC1, $\phi=10$} &
    {\tiny (v) Data3, MCMC2, $\phi=0$} &
    {\tiny (w) Data3, MCMC2, $\phi=1$} &
    {\tiny (x) Data3, MCMC2, $\phi=10$} \\
    %
    \includegraphics[scale=#2]{\imgdir/dataseed3-mcmcseed1-phi0.0#3/img/#1} &
    \includegraphics[scale=#2]{\imgdir/dataseed3-mcmcseed1-phi1.0#3/img/#1} &
    \includegraphics[scale=#2]{\imgdir/dataseed3-mcmcseed1-phi10.0#3/img/#1} &
    %
    \includegraphics[scale=#2]{\imgdir/dataseed3-mcmcseed2-phi0.0#3/img/#1} &
    \includegraphics[scale=#2]{\imgdir/dataseed3-mcmcseed2-phi1.0#3/img/#1} &
    \includegraphics[scale=#2]{\imgdir/dataseed3-mcmcseed2-phi10.0#3/img/#1} \\
    %
    {\tiny (y) Data3, MCMC3, $\phi=0$} &
    {\tiny (z) Data3, MCMC3, $\phi=1$} &
    {\tiny ($\alpha$) Data3, MCMC3, $\phi=10$} &
    & & \\
    %
    \includegraphics[scale=#2]{\imgdir/dataseed3-mcmcseed3-phi0.0#3/img/#1} &
    \includegraphics[scale=#2]{\imgdir/dataseed3-mcmcseed3-phi1.0#3/img/#1} &
    \includegraphics[scale=#2]{\imgdir/dataseed3-mcmcseed3-phi10.0#3/img/#1} \\
    & & \\
  \end{tabular}
}


% Title Settings
\title{Simulation Study 6.5.2}
\author{Arthur Lui}
\date{\today} % \date{} to set date to empty

% MAIN %
\begin{document}

\maketitle

% \tableofcontents \newpage % Comment to remove table of contents

% \section{Objective}\label{sec:objective}
% TODO

\section{Data Generation}\label{sec:data-generation}
Figure~\ref{fig:Z-true} and Figure~\ref{tab:W-true} respectively show the true
$Z$ and $W$ used to generate the simulated data in this study. Note the following:
\begin{itemize}
  \item Features (columns) 1, 2, and 3 are similar (maximum difference of 2).
    In Sample 1, where the first two features are rare / absent in one
    sample, and the third feature is abundant in both samples. We want to know
    if the first two features are prone to being absorbed into feature 3.
  \item Features 4 and 5 are similar (difference of 1). They are rare / absent
    in one sample and abundant in another. We are also curious if these two 
    features will be merged. Although, we don't expect them to. It is conceivable
    that they will only be ``selected'' in one sample, but not in another.
  \item Feature 6 is abundant in Sample 1 but not in Sample 2. We expect it to be
    recovered in $Z$ but it is conceivable that it will not be ``selected'' in sample
    2. It is distinct from other features.
  \item Feature 7 is rare in Sample 1, and absent in Sample 2. Nevertheless, we
    expect it to recovered in $Z$ because it is different from the other
    features.
\end{itemize}

\begin{figure}[H]
  \begin{center}  % 6 x 5
    \includegraphics[scale=.4]{\imgdir/dataseed_1/mcmcseed_1/scale_0/img/Z_true.pdf}
  \end{center}
  \caption{$Z^\true$}
  \label{fig:Z-true}
\end{figure}

% latex table generated in R 3.4.4 by xtable 1.8-3 package
% Wed Feb 12 14:53:27 2020
\begin{table}[ht]
  \Large
  \centering
  \begin{tabular}{rrrrrrrr}
    \hline
    & $k=1$ & $k=2$ & $k=3$ & $k=4$ & $k=5$ & $k=6$ & $k=7$ \\
    \hline
    Sample 1 & 0.04 & 0.05 & 0.39 & 0.00 & 0.06 & 0.43 & 0.03 \\
    Sample 2 & 0.00 & 0.05 & 0.54 & 0.20 & 0.14 & 0.07 & 0.00 \\
    \hline
  \end{tabular}
  \caption{$W^\true$}
  \label{tab:W-true}
\end{table}

\newpage
\section{Log likelihood (post-burn)}
\begin{figure}[H]
  \begin{center}  % 6 x 5
    \imgtemplate{loglike_postburn.pdf}{.15}
  \end{center}
\end{figure}

\newpage
\section{Posterior mean of $Z$}
\begin{figure}[H]
  \begin{center}  % 6 x 5
    \imgtemplate{Zmean.pdf}{.2}
  \end{center}
\end{figure}

\newpage
\section{Posterior estimate of $Z_1$}
\begin{figure}[H]
  \begin{center}  % 6 x 5
    \imgtemplate{Z1.pdf}{.15}
  \end{center}
\end{figure}

\newpage
\section{Posterior estimate of $Z_2$}
\begin{figure}[H]
  \begin{center}  % 6 x 5
    \imgtemplate{Z2.pdf}{.15}
  \end{center}
\end{figure}

\newpage
\section{Trace plots of $\mu^\star$}
\begin{figure}[H]
  \begin{center}  % 6 x 5
    \imgtemplate{mus_trace.pdf}{.2}
  \end{center}
\end{figure}

\newpage
\section{Posterior distributions of $\mu^\star$}
\begin{figure}[H]
  \begin{center}  % 6 x 5
    \imgtemplate{mus.pdf}{.2}
  \end{center}
\end{figure}

\newpage
\section{Trace plots of $\sigma^2$}
\begin{figure}[H]
  \begin{center}  % 6 x 5
    \imgtemplate{sig2_trace.pdf}{.2}
  \end{center}
\end{figure}

\newpage
\section{Posterior distribution of $\sigma^2$}
\begin{figure}[H]
  \begin{center}  % 6 x 5
    \imgtemplate{sig2.pdf}{.2}
  \end{center}
\end{figure}

\newpage
\section{Posterior distribution of $W$}
\begin{figure}[H]
  \begin{center}  % 6 x 5
    \imgtemplate{W.pdf}{.2}
  \end{center}
\end{figure}

\newpage
\section{Posterior distribution of $v$}
\begin{figure}[H]
  \begin{center}  % 6 x 5
    \imgtemplate{v.pdf}{.2}
  \end{center}
\end{figure}

% Uncomment if using bibliography:
% \bibliography{bib}
\end{document}
