% NOTE: sanitycheck
\newpage
\section{Log likelihood (post-burn)}
\begin{figure}[H]
  \begin{center}  % 6 x 5
    \imgtemplate{loglike_postburn.pdf}{\imscale}
  \end{center}
  \caption{Log-likelihood for 3000 iterations after burn-in of 6000 for each
  simulation setup. The first three columns are the runs with no missing
  data. The last three columns are the runs with missing data. The first and
  fourth, second and third, and third and fifth columns are run with $Z^1$,
  $Z^2$, and $Z^3$, respectively. Each row is run with a different $\phi$.
  This arrangement is used in subsequent figures. The mixing appears decent
  from these figures. We see that in some cases (b, e), swaps between chains
  resulted in very different states in the main chain.}
  \label{fig:ll}
\end{figure}

% NOTE: sanitycheck
\newpage
\section{Swap proportions}
\begin{figure}[H]
  \begin{center}  % 6 x 5
    \imgtemplate{swapprops.pdf}{\imscale}
  \end{center}
  \caption{These figures summarize proportion of instances that swaps were
  made when proposed between the various chains for each setup. It seems that
  swapping is more difficult when $\phi=0$, though I'm uncertain why that
  might be the case.}
  \label{fig:swapproprs}
\end{figure}

\newpage
\section{Posterior mean of $Z$}
\begin{figure}[H]
  \begin{center}  % 6 x 5
    \imgtemplate{Zmean.pdf}{\imscale}
  \end{center}
  % TODO: Write more?
  \caption{Posterior means of $Z$ for each run. When $\phi=0$, the number of
  recovered features is 0 to 1, for $Z^{1,\true}$. In contrast, number of
  recovered features when $\phi > 0$ is between 6 and 7, with the missing
  feature being feature 7. This is perhaps due to the added left-skew in the observed
  data, making it more feasible for features to be explained by mixture components
  with negative means. The $\sigma_2^i$ learned when $\phi=0$ tended to be
  much larger as well. Again, mixing may also be better when $\phi>0$ because 
  the exploration of different configurations of $Z$ is encourage. There is a
  tendency for $Z$ to be extremely sparse when $\phi=0$, due to the IBP prior.
  %
  For $Z^{2,\true}$, between 2 and 6 features were recovered when $\phi=0$.
  When $\phi > 0$, all features were recovered. But in $(n)$, 
  eight features were recovered, the extra feature being similar to one of the
  features. However, it's estimated abundance was less than 1\%.
  %
  For Data 3, when $\phi=0$, the only features that were recovered are
  features 3, 5, 6, and 7, which are the most representative features. i.e.
  Features 1 and 2, which are rare, are merged with feature 3, which is
  abundant and similar to the former features. When $\phi>0$, 6 to 7 features were
  recovered. The missing features were either feature 1 or 2, which is similar to 
  and merged with feature 3. Feature 7, though rare, was recovered, likely because
  it is distinct.}
  \label{fig:zmean}
\end{figure}

% NOTE: sanitycheck
\newpage
\section{Box plots for $p_i$}
\begin{figure}[H]
  \begin{center}  % 6 x 5
    \imgtemplate{p.pdf}{\imscale}
  \end{center}
  % TODO: Write more?
  \caption{The posterior mean of $p_i$ is usually between 0.2 and 0.4, and
  sometimes between 0.4 and 0.5. The spread is large.}
  \label{fig:ppost}
\end{figure}

\newpage
\section{Trace plots for $p_1$}
\begin{figure}[H]
  \begin{center}  % 6 x 5
    \imgtemplate{p1_trace.pdf}{\imscale}
  \end{center}
  % TODO: Write more?
  \caption{}
  \label{fig:p1trace}
\end{figure}

\newpage
\section{Trace plots for $p_2$}
\begin{figure}[H]
  \begin{center}  % 6 x 5
    \imgtemplate{p2_trace.pdf}{\imscale}
  \end{center}
  % TODO: Write more?
  \caption{}
  \label{fig:p2trace}
\end{figure}

% NOTE: sanitycheck
\newpage
\section{Trace plots for $\mu^\star$}
\begin{figure}[H]
  \begin{center}  % 6 x 5
    \imgtemplate{mus_trace.pdf}{\imscale}
  \end{center}
  \caption{Trace plots for $\mu^\star_z$. The trends are very similar within
  columns (i.e. for the same data, but different $\phi$). In (e), there is a
  jump in $\mu^\star_0$ which results from a swap proposed between the main
  chain and the next chain. In this run, helper chains usually did not swap
  with the main chain.}
  \label{fig:mus-trace}
\end{figure}

% NOTE: sanitycheck
\newpage
\section{Trace plots for $\sigma^2$}
\begin{figure}[H]
  \begin{center}  % 6 x 5
    \imgtemplate{sig2_trace.pdf}{\imscale}
  \end{center}
  % TODO: Write more?
  \caption{Trace plots for $\sigma^2$. The trends are very similar within
  columns (i.e. for the same data, but different $\phi$) when $\phi>0$.
  Again, In (e), there is a jump in $\sigma^2$ which results from a swap
  proposed between the main chain and the next chain. In this run, helper
  chains usually did not swap with the main chain.}
  \label{fig:sig2-trace}
\end{figure}

% sanitycheck
\newpage
\section{Posterior distributions for sample 1 marker 1}
\begin{figure}[H]
  \begin{center}  % 6 x 5
    \imgtemplate{dden/dden_i1_j1.pdf}{\imscale}
  \end{center}
  % TODO: Write more?
  \caption{Data fit is rather good.}
  \label{fig:ddi1j1}
\end{figure}

\newpage
\section{Posterior distributions for sample 1 marker 2}
\begin{figure}[H]
  \begin{center}  % 6 x 5
    \imgtemplate{dden/dden_i1_j2.pdf}{\imscale}
  \end{center}
  % TODO: Write more?
  \caption{}
  \label{fig:ddi1j2}
\end{figure}

\newpage
\section{Posterior distributions for sample 2 marker 1}
\begin{figure}[H]
  \begin{center}  % 6 x 5
    \imgtemplate{dden/dden_i2_j1.pdf}{\imscale}
  \end{center}
  % TODO: Write more?
  \caption{}
  \label{fig:ddi2j1}
\end{figure}

\newpage
\section{Posterior distributions for sample 2 marker 2}
\begin{figure}[H]
  \begin{center}  % 6 x 5
    \imgtemplate{dden/dden_i2_j2.pdf}{\imscale}
  \end{center}
  % TODO: Write more?
  \caption{}
  \label{fig:ddi2j2}
\end{figure}

\newpage
\section{Posterior distributions for sample 2 marker 6}
\begin{figure}[H]
  \begin{center}  % 6 x 5
    \imgtemplate{dden/dden_i1_j6.pdf}{\imscale}
  \end{center}
  % TODO: Write more?
  \caption{}
  \label{fig:ddi1j6}
\end{figure}

\newpage
\section{Posterior distributions for sample 2 marker 6}
\begin{figure}[H]
  \begin{center}  % 6 x 5
    \imgtemplate{dden/dden_i2_j6.pdf}{\imscale}
  \end{center}
  % TODO: Write more?
  \caption{}
  \label{fig:ddi2j6}
\end{figure}

\newpage
\section{Missing Mechanism}
\begin{figure}[H]
  \begin{center}  % 6 x 5
    \begin{tabular}{ccc}
      \hline \\
      & $Z_1$ & \\
      \includegraphics[scale=0.45]{\imgdir/pmiss0.0-phi0-zind1/img/missmech_1.pdf} &
      &
      \includegraphics[scale=0.45]{\imgdir/pmiss0.0-phi0-zind1/img/missmech_2.pdf} \\
      \hline \\
      & $Z_2$ & \\
      \includegraphics[scale=0.45]{\imgdir/pmiss0.0-phi0-zind2/img/missmech_1.pdf} &
      &
      \includegraphics[scale=0.45]{\imgdir/pmiss0.0-phi0-zind2/img/missmech_2.pdf} \\
      \hline \\
      & $Z_3$ & \\
      \includegraphics[scale=0.45]{\imgdir/pmiss0.0-phi0-zind3/img/missmech_1.pdf} &
      &
      \includegraphics[scale=0.45]{\imgdir/pmiss0.0-phi0-zind3/img/missmech_2.pdf} \\
      \hline
    \end{tabular}
  \end{center}
\label{fig:missmech}
\end{figure}
