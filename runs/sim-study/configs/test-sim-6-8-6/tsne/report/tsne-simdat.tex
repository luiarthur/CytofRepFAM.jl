%{{{1
\documentclass[11pt]{article} % 11-point font

\usepackage[margin=1in]{geometry} % set page to 1-inch margins
\usepackage{amsmath} % for math
\usepackage{amssymb} % like \Rightarrow
\setlength\parindent{0pt} % Suppresses the indentation of new paragraphs.

% Big display
\newcommand{\ds}{ \displaystyle }
% Parenthesis
\newcommand{\norm}[1]{\left\lVert#1\right\rVert}
\newcommand{\p}[1]{\left(#1\right)}
\newcommand{\bk}[1]{\left[#1\right]}
\newcommand{\bc}[1]{ \left\{#1\right\} }
\newcommand{\abs}[1]{ \left|#1\right| }
% Derivatives
\newcommand{\df}[2]{ \frac{d#1}{d#2} }
\newcommand{\ddf}[2]{ \frac{d^2#1}{d{#2}^2} }
\newcommand{\pd}[2]{ \frac{\partial#1}{\partial#2} }
\newcommand{\pdd}[2]{\frac{\partial^2#1}{\partial{#2}^2} }
% Distributions
\newcommand{\Normal}{ \text{Normal} }
\newcommand{\Beta}{ \text{Beta} }
\newcommand{\G}{ \text{Gamma} }
\newcommand{\InvGamma}{ \text{Inv-Gamma} }
\newcommand{\Uniform}{ \text{Uniform} }
% Statistics
\newcommand{\E}{ \text{E} }
\newcommand{\iid}{\overset{iid}{\sim}}
\newcommand{\ind}{\overset{ind}{\sim}}

% Graphics
\usepackage{graphicx}  % for figures
\usepackage{float} % Put figure exactly where I want [H]

% Uncomment if using bibliography
% Bibliography
% \usepackage{natbib}
% \bibliographystyle{plainnat}

% Adds settings for hyperlinks. (Mainly for table of contents.)
\usepackage{hyperref}
\hypersetup{
  pdfborder={0 0 0} % removes red box from links
}
%}}}1

\def\imgdir{../viz/img}
\newcommand{\imgtemplate}[2]{
  \begin{tabular}{ccccc}
    {\tiny (a) FlowSOM, Sample 1, Complete} &
    {\tiny (b) FlowSOM, Sample 2, Complete} &
    {\tiny (c) FlowSOM, Sample 1, Missing} &
    {\tiny (d) FlowSOM, Sample 2, Missing} \\
    \includegraphics[scale=#1]{\imgdir/tsne-flowsom1-pmiss0.0-phi0-zind#2.pdf} &
    \includegraphics[scale=#1]{\imgdir/tsne-flowsom2-pmiss0.0-phi0-zind#2.pdf} &
    \includegraphics[scale=#1]{\imgdir/tsne-flowsom1-pmiss0.6-phi0-zind#2.pdf} &
    \includegraphics[scale=#1]{\imgdir/tsne-flowsom2-pmiss0.6-phi0-zind#2.pdf} \\
    {\tiny (e) Mclust, Sample 1, Complete} &
    {\tiny (f) Mclust, Sample 2, Complete} &
    {\tiny (g) Mclust, Sample 1, Missing} &
    {\tiny (h) Mclust, Sample 2, Missing} \\
    \includegraphics[scale=#1]{\imgdir/tsne-mclust1-pmiss0.0-phi0-zind#2.pdf} &
    \includegraphics[scale=#1]{\imgdir/tsne-mclust2-pmiss0.0-phi0-zind#2.pdf} &
    \includegraphics[scale=#1]{\imgdir/tsne-mclust1-pmiss0.6-phi0-zind#2.pdf} &
    \includegraphics[scale=#1]{\imgdir/tsne-mclust2-pmiss0.6-phi0-zind#2.pdf} \\
    {\tiny (i) FAM, Sample 1, Complete} &
    {\tiny (j) FAM, Sample 2, Complete} &
    {\tiny (k) FAM, Sample 1, Missing} &
    {\tiny (l) FAM, Sample 2, Missing} \\
    \includegraphics[scale=#1]{\imgdir/tsne-rfam1-pmiss0.0-phi0-zind#2.pdf} &
    \includegraphics[scale=#1]{\imgdir/tsne-rfam2-pmiss0.0-phi0-zind#2.pdf} &
    \includegraphics[scale=#1]{\imgdir/tsne-rfam1-pmiss0.6-phi0-zind#2.pdf} &
    \includegraphics[scale=#1]{\imgdir/tsne-rfam2-pmiss0.6-phi0-zind#2.pdf} \\
    {\tiny (m) rFAM, Sample 1, Complete} &
    {\tiny (n) rFAM, Sample 2, Complete} &
    {\tiny (o) rFAM, Sample 1, Missing} &
    {\tiny (p) rFAM, Sample 2, Missing} \\
    \includegraphics[scale=#1]{\imgdir/tsne-rfam1-pmiss0.0-phi100-zind#2.pdf} &
    \includegraphics[scale=#1]{\imgdir/tsne-rfam2-pmiss0.0-phi100-zind#2.pdf} &
    \includegraphics[scale=#1]{\imgdir/tsne-rfam1-pmiss0.6-phi100-zind#2.pdf} &
    \includegraphics[scale=#1]{\imgdir/tsne-rfam2-pmiss0.6-phi100-zind#2.pdf} \\
    {\tiny (q) Truth, Sample 1, Complete} &
    {\tiny (r) Truth, Sample 2, Complete} &
    {\tiny (s) Truth, Sample 1, Missing} &
    {\tiny (t) Truth, Sample 2, Missing} \\
    \includegraphics[scale=#1]{\imgdir/tsne-true_labels1-pmiss0.0-phi0-zind#2.pdf} &
    \includegraphics[scale=#1]{\imgdir/tsne-true_labels2-pmiss0.0-phi0-zind#2.pdf} &
    \includegraphics[scale=#1]{\imgdir/tsne-true_labels1-pmiss0.6-phi0-zind#2.pdf} &
    \includegraphics[scale=#1]{\imgdir/tsne-true_labels2-pmiss0.6-phi0-zind#2.pdf} \\
  \end{tabular}
}

\def\true{\text{true}}

% Title Settings
\title{TSNE for Mclust and FlowSOM on Simulated Data}
\author{Arthur Lui}
\date{\today} % \date{} to set date to empty

% MAIN %
\begin{document}
\maketitle
For each dataset $(Z_1, Z_2, Z_3)$, there were two versions -- one with
complete data, one where missing values were present and imputed as in the
MCMC. So in total, there are 6 TSNE embeddings. Before computing the
embeddings, the (two) samples were concatenated into one matrix. \\

Mclust and FlowSOM were each set to use 10 clusters at most. And they each
learned the number of clusters. Mclust used the VII model. Different random
seeds were tried, and the most representative (and best) results were used.
The TSNE embeddings were colored by clusterings from the two clustering
methods. \\

In each page of the document, there are 20 graphs.  The rows vary by model
(FlowSOM, Mclust-VII, FAM, rFAM-$\phi=100$, truth).  The first two columns are
for the complete data; whereas the next two columns are for the data with
imputed missing values. In the case of the FAM and rFAM, the imputation was
done during the MCMC in order to obtain the clusters. But for more consistency,
the TSNEs are based on the same data, where the missing values were imputed
using the initialization technique for the FAM and rFAM. \\

Figure~\ref{fig:Z-true} and Table~\ref{tab:W-true} show, respectively, the
matrices $Z^\true$ and $W^\true$ used in the simulation study. \\

\begin{figure}[H]
  \begin{center}  % 6 x 5
    \begin{tabular}{ccc}
      \includegraphics[scale=0.25]{\imgdir/Z1.pdf} &
      \includegraphics[scale=0.25]{\imgdir/Z2.pdf} &
      \includegraphics[scale=0.25]{\imgdir/Z3.pdf} \\
      %
      (a) $Z^{1,\true}$ &
      (b) $Z^{2,\true}$ &
      (c) $Z^{3,\true}$ \\
      %
      column-distances of 2 bits &
      column-distances of 6 bits &
      mix of similar and distinct columns \\
    \end{tabular}
  \end{center}
  \caption{Simulation truth of $Z$ for three scenarios.}
  \label{fig:Z-true}
\end{figure}

% latex table generated in R 3.4.4 by xtable 1.8-3 package
% Wed Feb 12 14:53:27 2020
\begin{table}[ht]
  \centering
  \begin{tabular}{rrrrrrrr}
    \hline
    & $k=1$ & $k=2$ & $k=3$ & $k=4$ & $k=5$ & $k=6$ & $k=7$ \\
    \hline
    Sample 1 & 0.04 & 0.05 & 0.39 & 0.00 & 0.06 & 0.43 & 0.03 \\
    Sample 2 & 0.00 & 0.05 & 0.54 & 0.20 & 0.14 & 0.07 & 0.00 \\
    \hline
  \end{tabular}
  \caption{$W^\true$}
  \label{tab:W-true}
\end{table}


% \tableofcontents \newpage % Comment to remove table of contents

\section{TSNE for Mclust and FlowSOM for $Z_1$}  % 8 figs
\begin{figure}[H]
  \begin{center}
    \imgtemplate{0.2}{1}
  \end{center}
  \caption{In this dataset, the subpopulations are similar; every column pair
  distance is 1. The data is also noisy, making inference difficult. In the
  complete data setting, FlowSOM learns only 4 clusters (when there are 7).
  Mclust learns 7 clusters for the complete data. FlowSOM and Mclust learn 3
  and 10 clusters respectively for the missing-values dataset. Different
  random seeds were tried.}
\end{figure}
\newpage

\section{TSNE for Mclust and FlowSOM for $Z_2$}  % 8 figs
\begin{figure}[H]
  \begin{center}
    \imgtemplate{0.2}{2}
  \end{center}
  \caption{In this dataset, subpopulations are distinct; every column pair
  distance is 6. FlowSOM learns the 5 (of 6) clusters here, likely because
  the clusters are distinct enough. Mclust overestimates the number of
  clusters. It learns 8 and 10 clusters respectively in the complete and
  imputed datasets. FAM interprets noise as new subpopulations; whereas
  rFAM with $\phi=100$ learns the correct subpopulations.}
\end{figure}
\newpage

\section{TSNE for Mclust and FlowSOM for $Z_3$}  % 8 figs
\begin{figure}[H]
  \begin{center}
    \imgtemplate{0.2}{3}
  \end{center}
  \caption{In this dataset, distinct and similar subpopulations are present.
  FlowSOM learns fewer than 7 clusters and Mclust learns more than 7
  clusters. FlowSOM tends to recover high-level features, whereas Mclust can
  find smaller discrepancies between the cells, as shown by its breaking down
  larger groups into smaller clusters. FAM learns the correct number of
  subpopulations; while rFAM learns fewer than the true number of
  subpopulations. rFAM discourages similar subpopulations to be learned, and
  hence subpopulations that are similar to abundant subpopulations are
  merged.}
\end{figure}
\newpage

\end{document}
