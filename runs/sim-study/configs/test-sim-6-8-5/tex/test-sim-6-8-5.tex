%{{{1
\documentclass[10pt]{article} % 12-point font

\usepackage[margin=1cm]{geometry} % set page to 1-inch margins
\usepackage{amsmath} % for math
\usepackage{amssymb} % like \Rightarrow
\setlength\parindent{0pt} % Suppresses the indentation of new paragraphs.
\pagenumbering{gobble} % suppress page numbers

% Big display
\newcommand{\ds}{ \displaystyle }
% Parenthesis
\newcommand{\norm}[1]{\left\lVert#1\right\rVert}
\newcommand{\p}[1]{\left(#1\right)}
\newcommand{\bk}[1]{\left[#1\right]}
\newcommand{\bc}[1]{ \left\{#1\right\} }
\newcommand{\abs}[1]{ \left|#1\right| }
% Derivatives
\newcommand{\df}[2]{ \frac{d#1}{d#2} }
\newcommand{\ddf}[2]{ \frac{d^2#1}{d{#2}^2} }
\newcommand{\pd}[2]{ \frac{\partial#1}{\partial#2} }
\newcommand{\pdd}[2]{\frac{\partial^2#1}{\partial{#2}^2} }
% Distributions
\newcommand{\Normal}{ \text{Normal} }
\newcommand{\Beta}{ \text{Beta} }
\newcommand{\Gam}{ \text{Gamma} }
\newcommand{\InvGamma}{ \text{Inv-Gamma} }
\newcommand{\Uniform}{ \text{Uniform} }
\def\Dir{\text{Dirichlet}}
\def\TN{\text{TN}}
% Statistics
\newcommand{\E}{\text{E}}
\newcommand{\iid}{\overset{iid}{\sim}}
\newcommand{\ind}{\overset{ind}{\sim}}

% Graphics
\usepackage{graphicx}  % for figures
\usepackage{float} % Put figure exactly where I want [H]

% Uncomment if using bibliography
% Bibliography
\usepackage{natbib}
\bibliographystyle{plainnat}

% Adds settings for hyperlinks. (Mainly for table of contents.)
\usepackage{hyperref}
\hypersetup{
  pdfborder={0 0 0} % removes red box from links
}
%}}}1


% Macros for this project
\def\true{\text{TRUE}}
\usepackage[dvipsnames,usenames]{color}
\newcommand{\bbh}{\color{blue}\textbf}  % blue bold highlight
\newcommand{\bch}{\color{blue}\it}  % blue italics
\newcommand{\ech}{\color{black}\rm}
\def\imgdir{results}
\newcommand{\imgtemplate}[3]{
  \begin{tabular}{cccccc}
    {\tiny (a) Data1, MCMC1, $\phi=0$} &
    {\tiny (b) Data1, MCMC1, $\phi=1$}&
    {\tiny (c) Data1, MCMC1, $\phi=10$} &
    {\tiny (d) Data1, MCMC2, $\phi=0$} &
    {\tiny (e) Data1, MCMC2, $\phi=1$} &
    {\tiny (f) Data1, MCMC2, $\phi=10$} \\
    %
    \includegraphics[scale=#2]{\imgdir/dataseed1-mcmcseed1-phi0.0#3/img/#1} &
    \includegraphics[scale=#2]{\imgdir/dataseed1-mcmcseed1-phi1.0#3/img/#1} &
    \includegraphics[scale=#2]{\imgdir/dataseed1-mcmcseed1-phi10.0#3/img/#1} &
    %
    \includegraphics[scale=#2]{\imgdir/dataseed1-mcmcseed2-phi0.0#3/img/#1} &
    \includegraphics[scale=#2]{\imgdir/dataseed1-mcmcseed2-phi1.0#3/img/#1} &
    \includegraphics[scale=#2]{\imgdir/dataseed1-mcmcseed2-phi10.0#3/img/#1} \\
    %
    {\tiny (g) Data1, MCMC3, $\phi=0$} &
    {\tiny (h) Data1, MCMC3, $\phi=1$} &
    {\tiny (i) Data1, MCMC3, $\phi=10$} &
    {\tiny (j) Data2, MCMC1, $\phi=0$} &
    {\tiny (k) Data2, MCMC1, $\phi=1$} &
    {\tiny (l) Data2, MCMC1, $\phi=10$} \\
    %
    \includegraphics[scale=#2]{\imgdir/dataseed1-mcmcseed3-phi0.0#3/img/#1} &
    \includegraphics[scale=#2]{\imgdir/dataseed1-mcmcseed3-phi1.0#3/img/#1} &
    \includegraphics[scale=#2]{\imgdir/dataseed1-mcmcseed3-phi10.0#3/img/#1} &
    %
    \includegraphics[scale=#2]{\imgdir/dataseed2-mcmcseed1-phi0.0#3/img/#1} &
    \includegraphics[scale=#2]{\imgdir/dataseed2-mcmcseed1-phi1.0#3/img/#1} &
    \includegraphics[scale=#2]{\imgdir/dataseed2-mcmcseed1-phi10.0#3/img/#1} \\
    %
    {\tiny (m) Data2, MCMC2, $\phi=0$} &
    {\tiny (n) Data2, MCMC2, $\phi=1$} &
    {\tiny (o) Data2, MCMC2, $\phi=10$} &
    {\tiny (p) Data2, MCMC3, $\phi=0$} &
    {\tiny (q) Data2, MCMC3, $\phi=1$} &
    {\tiny (r) Data2, MCMC3, $\phi=10$} \\
    %
    \includegraphics[scale=#2]{\imgdir/dataseed2-mcmcseed2-phi0.0#3/img/#1} &
    \includegraphics[scale=#2]{\imgdir/dataseed2-mcmcseed2-phi1.0#3/img/#1} &
    \includegraphics[scale=#2]{\imgdir/dataseed2-mcmcseed2-phi10.0#3/img/#1} &
    %
    \includegraphics[scale=#2]{\imgdir/dataseed2-mcmcseed3-phi0.0#3/img/#1} &
    \includegraphics[scale=#2]{\imgdir/dataseed2-mcmcseed3-phi1.0#3/img/#1} &
    \includegraphics[scale=#2]{\imgdir/dataseed2-mcmcseed3-phi10.0#3/img/#1} \\
    %
    {\tiny (s) Data3, MCMC1, $\phi=0$} &
    {\tiny (t) Data3, MCMC1, $\phi=1$} &
    {\tiny (u) Data3, MCMC1, $\phi=10$} &
    {\tiny (v) Data3, MCMC2, $\phi=0$} &
    {\tiny (w) Data3, MCMC2, $\phi=1$} &
    {\tiny (x) Data3, MCMC2, $\phi=10$} \\
    %
    \includegraphics[scale=#2]{\imgdir/dataseed3-mcmcseed1-phi0.0#3/img/#1} &
    \includegraphics[scale=#2]{\imgdir/dataseed3-mcmcseed1-phi1.0#3/img/#1} &
    \includegraphics[scale=#2]{\imgdir/dataseed3-mcmcseed1-phi10.0#3/img/#1} &
    %
    \includegraphics[scale=#2]{\imgdir/dataseed3-mcmcseed2-phi0.0#3/img/#1} &
    \includegraphics[scale=#2]{\imgdir/dataseed3-mcmcseed2-phi1.0#3/img/#1} &
    \includegraphics[scale=#2]{\imgdir/dataseed3-mcmcseed2-phi10.0#3/img/#1} \\
    %
    {\tiny (y) Data3, MCMC3, $\phi=0$} &
    {\tiny (z) Data3, MCMC3, $\phi=1$} &
    {\tiny ($\alpha$) Data3, MCMC3, $\phi=10$} &
    & & \\
    %
    \includegraphics[scale=#2]{\imgdir/dataseed3-mcmcseed3-phi0.0#3/img/#1} &
    \includegraphics[scale=#2]{\imgdir/dataseed3-mcmcseed3-phi1.0#3/img/#1} &
    \includegraphics[scale=#2]{\imgdir/dataseed3-mcmcseed3-phi10.0#3/img/#1} \\
    & & \\
  \end{tabular}
}

\def\imscale{.16}

% Title Settings
\title{Simulation Study 6.8.4}
\author{Arthur Lui}
\date{\today} % \date{} to set date to empty

% MAIN %
\begin{document}

\maketitle

% \tableofcontents \newpage % Comment to remove table of contents

% \section{Objective}\label{sec:objective}
% TODO

\section{Data Generation}\label{sec:data-generation}
In this study, we generate data with two samples, each containing 2000 cells,
and 21 markers. Figure~\ref{fig:Z-true} and Table~\ref{tab:W-true}
respectively show the true $Z$ and $W$ used to generate the simulated data in
this study. Note the following:
\begin{itemize}
  \item Three $Z$'s, shown in Figure~\ref{fig:Z-true}, were used in this study.
  \begin{itemize}
    \item In $Z^{1,\true}$, each column is different from the other columns exactly
    by exactly two bits. In this sense, all columns are "close" to each
    other.
    \item In $Z^{2,\true}$, each column is different from the other columns exactly
    by exactly six bits. In this sense, all columns are "far" to each
    other.
    \item In $Z^{3,\true}$, the first three columns are similar. Columns 4 and 5 are 
    close to each other. And columns 6 and 7 are distinct from other columns. 
  \end{itemize}
  \item Figure~\ref{tab:W-true} shows the true $W$ used for this study. It
    contains feature abundances that are small in both samples, large in both
    samples, and large in only one sample. For example, feature $k=7$, 
    has an abundance of $3\%$ in sample 1, and is absent in feature 2. 
    Moreover, it has the smallest aggregate abundance across both samples. 
    Hence, we can expect it will be least likely recovered in the analysis. 
    Feature $k=3$, on the other hand is abundant in both samples, so we expect
    it to be easily recovered.
\end{itemize}

\begin{figure}[H]
  \begin{center}  % 6 x 5
    \begin{tabular}{ccc}
      \includegraphics[scale=0.4]{\imgdir/pmiss0.0-phi0-zind1/img/Z_true.pdf} &
      \includegraphics[scale=0.4]{\imgdir/pmiss0.0-phi0-zind2/img/Z_true.pdf} &
      \includegraphics[scale=0.4]{\imgdir/pmiss0.0-phi0-zind3/img/Z_true.pdf} \\
      %
      (a) $Z^{1,\true}$ &
      (b) $Z^{2,\true}$ &
      (c) $Z^{3,\true}$ \\
      %
      column-distances of 2 bits &
      column-distances of 6 bits &
      mix of similar and distinct columns \\
    \end{tabular}
  \end{center}
  \caption{Simulation truth of $Z$ for three scenarios.}
  \label{fig:Z-true}
\end{figure}

% latex table generated in R 3.4.4 by xtable 1.8-3 package
% Wed Feb 12 14:53:27 2020
\begin{table}[ht]
  \centering
  \begin{tabular}{rrrrrrrr}
    \hline
    & $k=1$ & $k=2$ & $k=3$ & $k=4$ & $k=5$ & $k=6$ & $k=7$ \\
    \hline
    Sample 1 & 0.04 & 0.05 & 0.39 & 0.00 & 0.06 & 0.43 & 0.03 \\
    Sample 2 & 0.00 & 0.05 & 0.54 & 0.20 & 0.14 & 0.07 & 0.00 \\
    \hline
  \end{tabular}
  \caption{$W^\true$}
  \label{tab:W-true}
\end{table}

\subsection{More on data generation / MCMC setup}
\begin{itemize}
  \item We generated $y_{inj}$ from skew Normal distributions. Specifically,
  \begin{itemize}
    % \item Location parameter: $(\mu_{0,i,j}^\star)^\true=-0.8 +
    %   \epsilon_{0,i,j}$, $\epsilon_{0,i,j} \sim \Uniform(-0.3, 0.3)$
    % \item Location parameter: $(\mu_{1,i,j}^\star)^\true=1.3 +
    %   \epsilon_{1,i,j}$, $\epsilon_{1,i,j} \sim \Uniform(-0.3, 0.3)$
    \item Location parameter: $(\mu_{0,i,j}^\star)^\true=-0.8 +
      \epsilon_{0,i,j}$, $\epsilon_{0,i,j} \sim \Uniform(-0.3, 0.3)$
    \item Location parameter: $(\mu_{1,i,j}^\star)^\true=1 +
      \epsilon_{1,i,j}$, $\epsilon_{1,i,j} \sim \Uniform(-0.3, 0.3)$
    \item Scale parameter: ($\sigma^2_i)^\true=0.7$, $N=(2000, 2000)$
    \item The skewness parameter $\zeta$ was set to -0.9. Note that $\zeta$
          corresponds to $\delta \in \p{-1, 1}$ in \cite{fruhwirth2010bayesian}.
  \end{itemize}
  % \item 6000 burn-in, followed by 3000 iterations (no thinning).
  \item 3000 burn-in, followed by 1000 iterations (no thinning).
  \item We also did weight-preserving parallel tempering (WPPT) with temperatures 
        $\bc{1, 1.003, 1.006, 1.01}$. Swaps between every pair of chains were
        proposed at each MCMC iteration.
  \item Computation time for each chain was was approximately 4 hours.
  \item 5\% of data in each sample are used to sample from trained prior, and
    $M=5$ was used.
  \item We did each analysis with $\phi \in \bc{0, 1, 25, 100}$
  \item The simulation study was done for each of the true $Z$'s in 
        Figure~\ref{fig:Z-true}.
  \item For each true $Z$, we generated a full data set with no missing values,
        and a data set with missing values (which is a subset of the former). Cells
        that truly express a marker were never marked as missing.
  \item In total, there were 24 runs, each run doing parallel tempering on 4
        temperatures.
  \item \textbf{Some Priors:}
  \begin{itemize}
    \item $w^\star_{i,k} \sim \Gam(\text{shape}=1, \text{rate}=1/2)$
    \item $p_c \sim \text{Beta}(1, 9)$
    \item $L_0=L_1=3$
    \item $\alpha \sim \Gam(0.1, 0.1)$
    \item $\delta_0 \sim \TN^-(1, 0.1)$
    \item $\delta_1 \sim \TN^+(1, 0.1)$
    \item $\eta_{z, i,j,\ell} \sim \Dir_{L_z}(1)$
    \item $\sigma^2_i \sim \InvGamma(3, 1)$
  \end{itemize}
\end{itemize}


\newpage
\section{Posterior estimate of $Z_1$}
\begin{figure}[H]
  \begin{center}  % 6 x 5
    \imgtemplate{Z1.pdf}{\imscale}
  \end{center}
  \caption{Posterior estimate of $Z_1$. Features were duplicated in all runs 
  where $\phi=0$. Zero to one features were not recovered. It appears except for
  (x), all features were recovered when $\phi > 0$.}
  \label{fig:z1est}
\end{figure}

\newpage
\section{Posterior estimate of $Z_2$}
\begin{figure}[H]
  \begin{center}  % 6 x 5
    \imgtemplate{Z2.pdf}{\imscale}
  \end{center}
  \caption{Posterior estimate of $Z_2$. Features were duplicated in most runs
  where $\phi=0$, and all features were recovered. For $\phi>0$, all runs
  recovered all features. In (h) and (k), trace amounts ($<$ 1\%) of a feature
  resembling one of the identified features was also learned.}
  \label{fig:z2est}
\end{figure}

\section{Centroids of Posterior estimate of $y_1$}
\begin{figure}[H]
  \begin{center}  % 6 x 5
    \imgtemplate{y1_centroid.pdf}{\imscale}
  \end{center}
  % TODO: Write more?
  \caption{}
  \label{fig:y1est}
\end{figure}

\newpage
\section{Centroids of Posterior estimate of $y_2$}
\begin{figure}[H]
  \begin{center}  % 6 x 5
    \imgtemplate{y2_centroid.pdf}{\imscale}
  \end{center}
  % TODO: Write more?
  \caption{}
  \label{fig:y2est}
\end{figure}

% NOTE: sanitycheck
\newpage
\section{Posterior estimate of $y_1$}
\begin{figure}[H]
  \begin{center}  % 6 x 5
    \imgtemplate{y1.pdf}{\imscale}
  \end{center}
  % TODO: Write more?
  \caption{}
  \label{fig:y1est}
\end{figure}
 
% \newpage
\section{Posterior estimate of $y_2$}
\begin{figure}[H]
  \begin{center}  % 6 x 5
    \imgtemplate{y2.pdf}{\imscale}
  \end{center}
  % TODO: Write more?
  \caption{}
  \label{fig:y2est}
\end{figure}

\newpage
\section{Posterior distributions for $\mu^\star$}
\begin{figure}[H]
  \begin{center}  % 6 x 5
    \imgtemplate{mus.pdf}{\imscale}
  \end{center}
  \caption{Box plots of $\mu^\star$.}
  \label{fig:mus}
\end{figure}


\newpage
\section{Posterior distributions for $\sigma^2$}
\begin{figure}[H]
  \begin{center}  % 6 x 5
    \imgtemplate{sig2.pdf}{\imscale}
  \end{center}
  % TODO: Write more?
  \caption{}
  \label{fig:sig2}
\end{figure}

% \newpage
\section{Log likelihood (post-burn)}
\begin{figure}[H]
  \begin{center}  % 6 x 5
    \imgtemplate{loglike_postburn.pdf}{\imscale}
  \end{center}
  \caption{Log-likelihood for 5000 iterations after burn-in of 10000 for each
  simulation setup.}
  \label{fig:ll}
\end{figure}

\newpage
\section{Swap proportions}
\begin{figure}[H]
  \begin{center}  % 6 x 5
    \imgtemplate{swapprops.pdf}{\imscale}
  \end{center}
  \caption{These figures summarize proportion of instances that swaps were made
  when proposed between the various chains for each setup. A noticeable amount
  of Swapping occurred between the chains. The colorbars on the right of each
  heatmap show that the highest proportion of swaps between any two chains is
  between 40\% and 60\%.}
  \label{fig:swapproprs}
\end{figure}


\newpage
\section{Posterior mean of $Z$}
\begin{figure}[H]
  \begin{center}  % 6 x 5
    \imgtemplate{Zmean.pdf}{\imscale}
  \end{center}
  % TODO: Write more?
  \caption{Posterior means of $Z$ for each run. This figure is included only
  to show that it should not be used, due to label switching which happens as
  a result of states-swapping between chains.}
  \label{fig:zmean}
\end{figure}

\newpage
\section{Box plots for $p_i$}
\begin{figure}[H]
  \begin{center}  % 6 x 5
    \imgtemplate{p.pdf}{\imscale}
  \end{center}
  % TODO: Write more?
  \caption{}
  \label{fig:ppost}
\end{figure}
 
\newpage
\section{Trace plots for $p_1$}
\begin{figure}[H]
  \begin{center}  % 6 x 5
    \imgtemplate{p1_trace.pdf}{\imscale}
  \end{center}
  % TODO: Write more?
  \caption{}
  \label{fig:p1trace}
\end{figure}
 
\newpage
\section{Trace plots for $p_2$}
\begin{figure}[H]
  \begin{center}  % 6 x 5
    \imgtemplate{p2_trace.pdf}{\imscale}
  \end{center}
  % TODO: Write more?
  \caption{}
  \label{fig:p2trace}
\end{figure}

% \newpage
% \section{Trace plots for $p_3$}
% \begin{figure}[H]
%   \begin{center}  % 6 x 5
%     \imgtemplate{p3_trace.pdf}{\imscale}
%   \end{center}
%   % TODO: Write more?
%   \caption{}
%   \label{fig:p3trace}
% \end{figure}

\newpage
\section{Box plots for $W$}
\begin{figure}[H]
  \begin{center}  % 6 x 5
    \imgtemplate{W.pdf}{\imscale}
  \end{center}
  \caption{Posterior distribution for $W$. This figure is not very useful
  to look at, due to label-switching.}
  \label{fig:W-post}
\end{figure}

\newpage
\section{Trace plots for $\mu^\star$}
\begin{figure}[H]
  \begin{center}  % 6 x 5
    \imgtemplate{mus_trace.pdf}{\imscale}
  \end{center}
  \caption{Trace plots for $\mu^\star_z$.}
  \label{fig:mus-trace}
\end{figure}
  
\newpage
\section{Trace plots for $\sigma^2$}
\begin{figure}[H]
  \begin{center}  % 6 x 5
    \imgtemplate{sig2_trace.pdf}{\imscale}
  \end{center}
  % TODO: Write more?
  \caption{Trace plots for $\sigma^2$.}
  \label{fig:sig2-trace}
\end{figure}

\newpage
\section{Posterior distributions for sample 1 marker 1}
\begin{figure}[H]
  \begin{center}  % 6 x 5
    \imgtemplate{dden/dden_i1_j1.pdf}{\imscale}
  \end{center}
  % TODO: Write more?
  \caption{}
  \label{fig:ddi1j1}
\end{figure}

\newpage
\section{Posterior distributions for sample 1 marker 2}
\begin{figure}[H]
  \begin{center}  % 6 x 5
    \imgtemplate{dden/dden_i1_j2.pdf}{\imscale}
  \end{center}
  % TODO: Write more?
  \caption{}
  \label{fig:ddi1j2}
\end{figure}

\newpage
\section{Posterior distributions for sample 2 marker 1}
\begin{figure}[H]
  \begin{center}  % 6 x 5
    \imgtemplate{dden/dden_i2_j1.pdf}{\imscale}
  \end{center}
  % TODO: Write more?
  \caption{}
  \label{fig:ddi2j1}
\end{figure}

\newpage
\section{Posterior distributions for sample 2 marker 2}
\begin{figure}[H]
  \begin{center}  % 6 x 5
    \imgtemplate{dden/dden_i2_j2.pdf}{\imscale}
  \end{center}
  % TODO: Write more?
  \caption{}
  \label{fig:ddi2j2}
\end{figure}

% \newpage
% \section{Posterior distributions for sample 3 marker 1}
% \begin{figure}[H]
%   \begin{center}  % 6 x 5
%     \imgtemplate{dden/dden_i3_j1.pdf}{\imscale}
%   \end{center}
%   % TODO: Write more?
%   \caption{}
%   \label{fig:ddi1j6}
% \end{figure}
% 
% \newpage
% \section{Posterior distributions for sample 3 marker 2}
% \begin{figure}[H]
%   \begin{center}  % 6 x 5
%     \imgtemplate{dden/dden_i3_j2.pdf}{\imscale}
%   \end{center}
%   % TODO: Write more?
%   \caption{}
%   \label{fig:ddi2j6}
% \end{figure}

\newpage
\section{Missing Mechanisms}
\begin{figure}[H]
  \begin{center}  % 6 x 5
    \begin{tabular}{ccc}
      \includegraphics[scale=0.35]{\imgdir/phi0/img/missmech_1.pdf} &
      \includegraphics[scale=0.35]{\imgdir/phi0/img/missmech_2.pdf} &
      % \includegraphics[scale=0.35]{\imgdir/phi0/img/missmech_3.pdf} \\
    \end{tabular}
  \end{center}
  \caption{}
\label{fig:missmech}
\end{figure}


% Uncomment if using bibliography:
\bibliography{sim}

\end{document}

