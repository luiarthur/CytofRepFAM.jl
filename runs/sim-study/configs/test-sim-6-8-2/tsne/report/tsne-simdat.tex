%{{{1
\documentclass[11pt]{article} % 11-point font

\usepackage[margin=1in]{geometry} % set page to 1-inch margins
\usepackage{amsmath} % for math
\usepackage{amssymb} % like \Rightarrow
\setlength\parindent{0pt} % Suppresses the indentation of new paragraphs.

% Big display
\newcommand{\ds}{ \displaystyle }
% Parenthesis
\newcommand{\norm}[1]{\left\lVert#1\right\rVert}
\newcommand{\p}[1]{\left(#1\right)}
\newcommand{\bk}[1]{\left[#1\right]}
\newcommand{\bc}[1]{ \left\{#1\right\} }
\newcommand{\abs}[1]{ \left|#1\right| }
% Derivatives
\newcommand{\df}[2]{ \frac{d#1}{d#2} }
\newcommand{\ddf}[2]{ \frac{d^2#1}{d{#2}^2} }
\newcommand{\pd}[2]{ \frac{\partial#1}{\partial#2} }
\newcommand{\pdd}[2]{\frac{\partial^2#1}{\partial{#2}^2} }
% Distributions
\newcommand{\Normal}{ \text{Normal} }
\newcommand{\Beta}{ \text{Beta} }
\newcommand{\G}{ \text{Gamma} }
\newcommand{\InvGamma}{ \text{Inv-Gamma} }
\newcommand{\Uniform}{ \text{Uniform} }
% Statistics
\newcommand{\E}{ \text{E} }
\newcommand{\iid}{\overset{iid}{\sim}}
\newcommand{\ind}{\overset{ind}{\sim}}

% Graphics
\usepackage{graphicx}  % for figures
\usepackage{float} % Put figure exactly where I want [H]

% Uncomment if using bibliography
% Bibliography
% \usepackage{natbib}
% \bibliographystyle{plainnat}

% Adds settings for hyperlinks. (Mainly for table of contents.)
\usepackage{hyperref}
\hypersetup{
  pdfborder={0 0 0} % removes red box from links
}
%}}}1

\def\imgdir{../viz/img}
\newcommand{\imgtemplate}[2]{
  \begin{tabular}{ccc}
    (a) FlowSOM, Sample 1, Complete &
    (b) Mclust, Sample 1, Complete &
    (c) Truth, Sample 1, Complete \\
    \includegraphics[scale=#1]{\imgdir/tsne-flowsom1-pmiss0.0-phi0.0-zind#2.pdf} &
    \includegraphics[scale=#1]{\imgdir/tsne-mclust1-pmiss0.0-phi0.0-zind#2.pdf} &
    \includegraphics[scale=#1]{\imgdir/tsne-true_labels1-pmiss0.0-phi0.0-zind#2.pdf} \\
    (d) FlowSOM, Sample 2, Complete &
    (e) Mclust, Sample 2, Complete &
    (f) Truth, Sample 2, Complete \\
    \includegraphics[scale=#1]{\imgdir/tsne-flowsom2-pmiss0.0-phi0.0-zind#2.pdf} &
    \includegraphics[scale=#1]{\imgdir/tsne-mclust2-pmiss0.0-phi0.0-zind#2.pdf} &
    \includegraphics[scale=#1]{\imgdir/tsne-true_labels2-pmiss0.0-phi0.0-zind#2.pdf} \\
    (g) FlowSOM, Sample 1, Missing &
    (h) Mclust, Sample 1, Missing &
    (i) Truth, Sample 1, Missing \\
    \includegraphics[scale=#1]{\imgdir/tsne-flowsom1-pmiss0.6-phi0.0-zind#2.pdf} &
    \includegraphics[scale=#1]{\imgdir/tsne-mclust1-pmiss0.6-phi0.0-zind#2.pdf} &
    \includegraphics[scale=#1]{\imgdir/tsne-true_labels1-pmiss0.6-phi0.0-zind#2.pdf} \\
    (j) FlowSOM, Sample 2, Missing &
    (k) Mclust, Sample 2, Missing &
    (l) Truth, Sample 2, Missing \\
    \includegraphics[scale=#1]{\imgdir/tsne-flowsom2-pmiss0.6-phi0.0-zind#2.pdf} &
    \includegraphics[scale=#1]{\imgdir/tsne-mclust2-pmiss0.6-phi0.0-zind#2.pdf} &
    \includegraphics[scale=#1]{\imgdir/tsne-true_labels2-pmiss0.6-phi0.0-zind#2.pdf} \\
  \end{tabular}
}

\def\true{\text{true}}

% Title Settings
\title{TSNE for Mclust and FlowSOM on Simulated Data}
\author{Arthur Lui}
\date{\today} % \date{} to set date to empty

% MAIN %
\begin{document}
\maketitle
For each dataset $(Z_1, Z_2, Z_3)$, there were two versions -- one with
complete data, one where missing values were present and imputed as in the
MCMC. So in total, there are 6 TSNE embeddings. Before computing the
embeddings, the (two) samples were concatenated into one matrix. \\

Mclust and FlowSOM were each set to use 10 clusters at most. And they each
learned the number of clusters. Mclust used the VII model. Different random
seeds were tried, and the most representative (and best) results were used.
The TSNE embeddings were colored by clusterings from the two clustering
methods. \\

In each page of the document, there are 12 graphs. The left, middle, and
right columns are the TSNEs marked by the clusterings from FlowSOM, Mclust,
and the true labels respectively. Rows 1 and 3 are for sample 1; whereas
rows 2 and 4 are for sample 2. The first two rows are with complete data;
while the next rows are with imputed missing data. \\

Figure~\ref{fig:Z-true} and Table~\ref{tab:W-true} show, respectively, the
matrices $Z^\true$ and $W^\true$ used in the simulation study. \\

\begin{figure}[H]
  \begin{center}  % 6 x 5
    \begin{tabular}{ccc}
      \includegraphics[scale=0.25]{\imgdir/Z1.pdf} &
      \includegraphics[scale=0.25]{\imgdir/Z2.pdf} &
      \includegraphics[scale=0.25]{\imgdir/Z3.pdf} \\
      %
      (a) $Z^{1,\true}$ &
      (b) $Z^{2,\true}$ &
      (c) $Z^{3,\true}$ \\
      %
      column-distances of 2 bits &
      column-distances of 6 bits &
      mix of similar and distinct columns \\
    \end{tabular}
  \end{center}
  \caption{Simulation truth of $Z$ for three scenarios.}
  \label{fig:Z-true}
\end{figure}

% latex table generated in R 3.4.4 by xtable 1.8-3 package
% Wed Feb 12 14:53:27 2020
\begin{table}[ht]
  \centering
  \begin{tabular}{rrrrrrrr}
    \hline
    & $k=1$ & $k=2$ & $k=3$ & $k=4$ & $k=5$ & $k=6$ & $k=7$ \\
    \hline
    Sample 1 & 0.04 & 0.05 & 0.39 & 0.00 & 0.06 & 0.43 & 0.03 \\
    Sample 2 & 0.00 & 0.05 & 0.54 & 0.20 & 0.14 & 0.07 & 0.00 \\
    \hline
  \end{tabular}
  \caption{$W^\true$}
  \label{tab:W-true}
\end{table}


% \tableofcontents \newpage % Comment to remove table of contents

\section{TSNE for Mclust and FlowSOM for $Z_1$}  % 8 figs
\begin{figure}[H]
  \begin{center}
    \imgtemplate{0.3}{1}
  \end{center}
  \caption{The clusterings are very reasonable. But in the complete data
  setting, FlowSOM learns only 5 clusters (when there are 7). Mclust learns
  8 clusters for the complete data. FlowSOM and Mclust learn 9 and 10 clusters
  respectively for the missing-values dataset. Different random seeds were
  tried. For FlowSOM, 5 clusters was the maximum number of clusters learned
  in the complete-data scenario.}
\end{figure}
\newpage

\section{TSNE for Mclust and FlowSOM for $Z_2$}  % 8 figs
\begin{figure}[H]
  \begin{center}
    \imgtemplate{0.3}{2}
  \end{center}
  \caption{FlowSOM learns the right number of clusters here. Probably because
  the clusters are distinct enough. Mclust struggles to find the correct
  number of clusters. It learns 8 and 10 clusters respectively in the complete
  and imputed datasets. It appears the largest group is being broken down.}
\end{figure}
\newpage

\section{TSNE for Mclust and FlowSOM for $Z_3$}  % 8 figs
\begin{figure}[H]
  \begin{center}
    \imgtemplate{0.3}{3}
  \end{center}
  \caption{In this scenario, FlowSOM learns fewer than 7 clusters and Mclust
  learns more than 7 clusters. FlowSOM tends to recover high-level features,
  whereas Mclust can find smaller discrepancies between the cells, as shown by
  its breaking down larger groups into smaller clusters.}
\end{figure}
\newpage

\end{document}
