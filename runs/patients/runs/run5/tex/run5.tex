%{{{1
\documentclass[11pt]{article} % 12-point font

\usepackage[margin=1in]{geometry} % set page to 1-inch margins
\usepackage{amsmath} % for math
\usepackage{amssymb} % like \Rightarrow
\setlength\parindent{0pt} % Suppresses the indentation of new paragraphs.
\pagenumbering{gobble} % suppress page numbers

% Big display
\newcommand{\ds}{ \displaystyle }
% Parenthesis
\newcommand{\norm}[1]{\left\lVert#1\right\rVert}
\newcommand{\p}[1]{\left(#1\right)}
\newcommand{\bk}[1]{\left[#1\right]}
\newcommand{\bc}[1]{ \left\{#1\right\} }
\newcommand{\abs}[1]{ \left|#1\right| }
% Derivatives
\newcommand{\df}[2]{ \frac{d#1}{d#2} }
\newcommand{\ddf}[2]{ \frac{d^2#1}{d{#2}^2} }
\newcommand{\pd}[2]{ \frac{\partial#1}{\partial#2} }
\newcommand{\pdd}[2]{\frac{\partial^2#1}{\partial{#2}^2} }
% Distributions
\newcommand{\Normal}{ \text{Normal} }
\newcommand{\Beta}{ \text{Beta} }
\newcommand{\Gam}{ \text{Gamma} }
\newcommand{\InvGamma}{ \text{Inv-Gamma} }
\newcommand{\Uniform}{ \text{Uniform} }
\def\Dir{\text{Dirichlet}}
\def\TN{\text{TN}}
% Statistics
\newcommand{\E}{\text{E}}
\newcommand{\iid}{\overset{iid}{\sim}}
\newcommand{\ind}{\overset{ind}{\sim}}

% Graphics
\usepackage{graphicx}  % for figures
\usepackage{float} % Put figure exactly where I want [H]

% Uncomment if using bibliography
% Bibliography
\usepackage{natbib}
\bibliographystyle{plainnat}

% Adds settings for hyperlinks. (Mainly for table of contents.)
\usepackage{hyperref}
\hypersetup{
  pdfborder={0 0 0} % removes red box from links
}
%}}}1


% Macros for this project
\def\true{\text{TRUE}}
\usepackage[dvipsnames,usenames]{color}
\newcommand{\bbh}{\color{blue}\textbf}  % blue bold highlight
\newcommand{\bch}{\color{blue}\it}  % blue italics
\newcommand{\ech}{\color{black}\rm}
\def\imgdir{results}
\newcommand{\imgtemplate}[3]{
  \begin{tabular}{cccccc}
    {\tiny (a) Data1, MCMC1, $\phi=0$} &
    {\tiny (b) Data1, MCMC1, $\phi=1$}&
    {\tiny (c) Data1, MCMC1, $\phi=10$} &
    {\tiny (d) Data1, MCMC2, $\phi=0$} &
    {\tiny (e) Data1, MCMC2, $\phi=1$} &
    {\tiny (f) Data1, MCMC2, $\phi=10$} \\
    %
    \includegraphics[scale=#2]{\imgdir/dataseed1-mcmcseed1-phi0.0#3/img/#1} &
    \includegraphics[scale=#2]{\imgdir/dataseed1-mcmcseed1-phi1.0#3/img/#1} &
    \includegraphics[scale=#2]{\imgdir/dataseed1-mcmcseed1-phi10.0#3/img/#1} &
    %
    \includegraphics[scale=#2]{\imgdir/dataseed1-mcmcseed2-phi0.0#3/img/#1} &
    \includegraphics[scale=#2]{\imgdir/dataseed1-mcmcseed2-phi1.0#3/img/#1} &
    \includegraphics[scale=#2]{\imgdir/dataseed1-mcmcseed2-phi10.0#3/img/#1} \\
    %
    {\tiny (g) Data1, MCMC3, $\phi=0$} &
    {\tiny (h) Data1, MCMC3, $\phi=1$} &
    {\tiny (i) Data1, MCMC3, $\phi=10$} &
    {\tiny (j) Data2, MCMC1, $\phi=0$} &
    {\tiny (k) Data2, MCMC1, $\phi=1$} &
    {\tiny (l) Data2, MCMC1, $\phi=10$} \\
    %
    \includegraphics[scale=#2]{\imgdir/dataseed1-mcmcseed3-phi0.0#3/img/#1} &
    \includegraphics[scale=#2]{\imgdir/dataseed1-mcmcseed3-phi1.0#3/img/#1} &
    \includegraphics[scale=#2]{\imgdir/dataseed1-mcmcseed3-phi10.0#3/img/#1} &
    %
    \includegraphics[scale=#2]{\imgdir/dataseed2-mcmcseed1-phi0.0#3/img/#1} &
    \includegraphics[scale=#2]{\imgdir/dataseed2-mcmcseed1-phi1.0#3/img/#1} &
    \includegraphics[scale=#2]{\imgdir/dataseed2-mcmcseed1-phi10.0#3/img/#1} \\
    %
    {\tiny (m) Data2, MCMC2, $\phi=0$} &
    {\tiny (n) Data2, MCMC2, $\phi=1$} &
    {\tiny (o) Data2, MCMC2, $\phi=10$} &
    {\tiny (p) Data2, MCMC3, $\phi=0$} &
    {\tiny (q) Data2, MCMC3, $\phi=1$} &
    {\tiny (r) Data2, MCMC3, $\phi=10$} \\
    %
    \includegraphics[scale=#2]{\imgdir/dataseed2-mcmcseed2-phi0.0#3/img/#1} &
    \includegraphics[scale=#2]{\imgdir/dataseed2-mcmcseed2-phi1.0#3/img/#1} &
    \includegraphics[scale=#2]{\imgdir/dataseed2-mcmcseed2-phi10.0#3/img/#1} &
    %
    \includegraphics[scale=#2]{\imgdir/dataseed2-mcmcseed3-phi0.0#3/img/#1} &
    \includegraphics[scale=#2]{\imgdir/dataseed2-mcmcseed3-phi1.0#3/img/#1} &
    \includegraphics[scale=#2]{\imgdir/dataseed2-mcmcseed3-phi10.0#3/img/#1} \\
    %
    {\tiny (s) Data3, MCMC1, $\phi=0$} &
    {\tiny (t) Data3, MCMC1, $\phi=1$} &
    {\tiny (u) Data3, MCMC1, $\phi=10$} &
    {\tiny (v) Data3, MCMC2, $\phi=0$} &
    {\tiny (w) Data3, MCMC2, $\phi=1$} &
    {\tiny (x) Data3, MCMC2, $\phi=10$} \\
    %
    \includegraphics[scale=#2]{\imgdir/dataseed3-mcmcseed1-phi0.0#3/img/#1} &
    \includegraphics[scale=#2]{\imgdir/dataseed3-mcmcseed1-phi1.0#3/img/#1} &
    \includegraphics[scale=#2]{\imgdir/dataseed3-mcmcseed1-phi10.0#3/img/#1} &
    %
    \includegraphics[scale=#2]{\imgdir/dataseed3-mcmcseed2-phi0.0#3/img/#1} &
    \includegraphics[scale=#2]{\imgdir/dataseed3-mcmcseed2-phi1.0#3/img/#1} &
    \includegraphics[scale=#2]{\imgdir/dataseed3-mcmcseed2-phi10.0#3/img/#1} \\
    %
    {\tiny (y) Data3, MCMC3, $\phi=0$} &
    {\tiny (z) Data3, MCMC3, $\phi=1$} &
    {\tiny ($\alpha$) Data3, MCMC3, $\phi=10$} &
    & & \\
    %
    \includegraphics[scale=#2]{\imgdir/dataseed3-mcmcseed3-phi0.0#3/img/#1} &
    \includegraphics[scale=#2]{\imgdir/dataseed3-mcmcseed3-phi1.0#3/img/#1} &
    \includegraphics[scale=#2]{\imgdir/dataseed3-mcmcseed3-phi10.0#3/img/#1} \\
    & & \\
  \end{tabular}
}

\def\imscale{.5}

% Title Settings
\title{Patients Data Analysis with r-FAM and $K=20$}
\author{Arthur Lui}
\date{\today} % \date{} to set date to empty

% MAIN %
\begin{document}

\maketitle

% \tableofcontents \newpage % Comment to remove table of contents

% \section{Objective}\label{sec:objective}
% TODO

\section{Data}\label{sec:data}
The data for this study consists of transformed marker expression levels
obtained from the blood samples of three subjects. Each of the three samples
belongs to a different subject. The number of cells in the samples are (4677,
1367, 12843). Thirty-two markers were studied in all samples and are listed
in Table~\ref{tab:markers}. To remove overly-extreme values in the analysis,
the data were further processed so that cells where at least one transformed
marker expression level was greater than 4 or less than -7 were excluded. The
final cell-counts in each sample was (4556, 1308, 12256).

% THE 32 MARKERS:
% 2B4 2DL1 2DL3 2DS4 3DL1 CCR7 CD158B CD16 CD25 CD27 CD57 CD62L CD8 CD94 CKIT
% DNAM1 EOMES GRA GRB KLRG1 LFA1 NKG2A NKG2C NKG2D NKP30 PERFORIN SIGLEC7 SYK
% TBET TIGIT TRAIL ZAP70

\begin{table}[H]
  \begin{center}
    \begin{tabular}{|l|l|l|l|}
      \hline
      1: 2B4    &  9: CD25   &  17: EOMES & 25: NKP30    \\
      2: 2DL1   &  10: CD27  &  18: GRA   & 26: PERFORIN \\
      3: 2DL3   &  11: CD57  &  19: GRB   & 27: SIGLEC7  \\
      4: 2DS4   &  12: CD62L &  20: KLRG1 & 28: SYK      \\   
      5: 3DL1   &  13: CD8   &  21: LFA1  & 29: TBET     \\
      6: CCR7   &  14: CD94  &  22: NKG2A & 30: TIGIT    \\ 
      7: CD158B &  15: CKIT  &  23: NKG2C & 31: TRAIL    \\ 
      8: CD16   &  16: DNAM1 &  24: NKG2D & 32: ZAP70    \\ 
      \hline
    \end{tabular} 
  \end{center}
  \caption{Index for markers referenced in this study.}
  \label{tab:markers}
\end{table}

\subsection{MCMC setup}
\begin{itemize}
  \item 10000 burn-in. 5000 samples were collected (no thinning).
  \item We also did weight-preserving parallel tempering (WPPT) with temperatures 
        $\bc{1, 1.003, 1.006, 1.01}$. Swaps between every pair of chains were
        proposed at each MCMC iteration.
  \item We fit 4 models, each with a different value for $\phi$. The values
        for $\phi$ were (0, 1, 10, 25).
  \item The average computation time for each model was was approximately 169 hours.
  \item 5\% of data in each sample are used to sample from trained prior, and
    $M=5$ was used.
  \item \textbf{Some Priors:}
  \begin{itemize}
    \item $w^\star_{i,k} \sim \Gam(\text{shape}=1, \text{rate}=1/2)$
    \item $p_c \sim \text{Beta}(1, 9)$
    \item $L_0=6$, $L_1=3$
    \item $\alpha \sim \Gam(1, 1)$
    \item $\delta_0 \sim \TN^-(1, 0.1)$
    \item $\delta_1 \sim \TN^+(1, 0.1)$
    \item $\eta_{z, i,j,\ell} \sim \Dir_{L_z}(1)$
    \item $\sigma^2_i \sim \InvGamma(3, 1)$
  \end{itemize}
\end{itemize}

\newpage
\section{Log likelihood (post-burn)}
\begin{figure}[H]
  \begin{center}  % 6 x 5
    \imgtemplate{loglike_postburn.pdf}{\imscale}
  \end{center}
  \caption{Log-likelihood for 5000 iterations after burn-in of 1000 for each
  simulation setup.}
  \label{fig:ll}
\end{figure}

\newpage
\section{Swap proportions}
\begin{figure}[H]
  \begin{center}  % 6 x 5
    \imgtemplate{swapprops.pdf}{\imscale}
  \end{center}
  \caption{These figures summarize proportion of instances that swaps were
  made when proposed between the various chains for each setup. Swapping rarely
  happened between any two chains.}
  \label{fig:swapproprs}
\end{figure}


\newpage
\section{Posterior mean of $Z$}
\begin{figure}[H]
  \begin{center}  % 6 x 5
    \imgtemplate{Zmean.pdf}{\imscale}
  \end{center}
  % TODO: Write more?
  \caption{Posterior means of $Z$ for each run. Z  }
  \label{fig:zmean}
\end{figure}
 
\newpage
\section{Posterior estimate of $Z_1$}
\begin{figure}[H]
  \begin{center}  % 6 x 5
    \imgtemplate{Z1.pdf}{\imscale}
  \end{center}
  \caption{Posterior estimate of $Z_1$.}
  \label{fig:z1est}
\end{figure}

\newpage
\section{Posterior estimate of $Z_2$}
\begin{figure}[H]
  \begin{center}  % 6 x 5
    \imgtemplate{Z2.pdf}{\imscale}
  \end{center}
  \caption{Posterior estimate of $Z_2$.}
  \label{fig:z2est}
\end{figure}

\newpage
\section{Posterior estimate of $Z_3$}
\begin{figure}[H]
  \begin{center}  % 2 x 2
    \imgtemplate{Z3.pdf}{\imscale}
  \end{center}
  \caption{Posterior estimate of $Z_3$.}
  \label{fig:z3est}
\end{figure}


\newpage
\section{Posterior estimate of $y_1$}
\begin{figure}[H]
  \begin{center}  % 6 x 5
    \imgtemplate{y1.pdf}{\imscale}
  \end{center}
  % TODO: Write more?
  \caption{}
  \label{fig:y1est}
\end{figure}

\newpage
\section{Posterior estimate of $y_2$}
\begin{figure}[H]
  \begin{center}  % 6 x 5
    \imgtemplate{y2.pdf}{\imscale}
  \end{center}
  % TODO: Write more?
  \caption{}
  \label{fig:y2est}
\end{figure}

\newpage
\section{Posterior estimate of $y_3$}
\begin{figure}[H]
  \begin{center}  % 6 x 5
    \imgtemplate{y3.pdf}{\imscale}
  \end{center}
  % TODO: Write more?
  \caption{}
  \label{fig:y3est}
\end{figure}



\newpage
\section{Box plots for $p_i$}
\begin{figure}[H]
  \begin{center}  % 6 x 5
    \imgtemplate{p.pdf}{\imscale}
  \end{center}
  % TODO: Write more?
  \caption{}
  \label{fig:ppost}
\end{figure}
 
\newpage
\section{Trace plots for $p_1$}
\begin{figure}[H]
  \begin{center}  % 6 x 5
    \imgtemplate{p1_trace.pdf}{\imscale}
  \end{center}
  % TODO: Write more?
  \caption{}
  \label{fig:p1trace}
\end{figure}
 
\newpage
\section{Trace plots for $p_2$}
\begin{figure}[H]
  \begin{center}  % 6 x 5
    \imgtemplate{p2_trace.pdf}{\imscale}
  \end{center}
  % TODO: Write more?
  \caption{}
  \label{fig:p2trace}
\end{figure}

\newpage
\section{Trace plots for $p_3$}
\begin{figure}[H]
  \begin{center}  % 6 x 5
    \imgtemplate{p3_trace.pdf}{\imscale}
  \end{center}
  % TODO: Write more?
  \caption{}
  \label{fig:p3trace}
\end{figure}

\newpage
\section{Posterior distribution for $R_i$}
\begin{figure}[H]
  \begin{center}  % 6 x 5
    \imgtemplate{Rcounts.pdf}{.4}
  \end{center}
  \caption{Posterior distribution for $R_i$.}
  \label{fig:W-post}
\end{figure}
 
\newpage
\section{Box plots for $W$}
\begin{figure}[H]
  \begin{center}  % 6 x 5
    \imgtemplate{W.pdf}{\imscale}
  \end{center}
  \caption{Posterior distribution for $W$.}
  \label{fig:W-post}
\end{figure}
 
\newpage
\section{Trace plots for $\mu^\star$}
\begin{figure}[H]
  \begin{center}  % 6 x 5
    \imgtemplate{mus_trace.pdf}{\imscale}
  \end{center}
  \caption{Trace plots for $\mu^\star_z$.}
  \label{fig:mus-trace}
\end{figure}
 
\newpage
\section{Posterior distributions for $\mu^\star$}
\begin{figure}[H]
  \begin{center}  % 6 x 5
    \imgtemplate{mus.pdf}{\imscale}
  \end{center}
  \caption{Box plots of $\mu^\star$.}
  \label{fig:mus}
\end{figure}


\newpage
\section{Trace plots for $\sigma^2$}
\begin{figure}[H]
  \begin{center}  % 6 x 5
    \imgtemplate{sig2_trace.pdf}{\imscale}
  \end{center}
  % TODO: Write more?
  \caption{Trace plots for $\sigma^2$.}
  \label{fig:sig2-trace}
\end{figure}

\newpage
\section{Posterior distributions for $\sigma^2$}
\begin{figure}[H]
  \begin{center}  % 6 x 5
    \imgtemplate{sig2.pdf}{\imscale}
  \end{center}
  % TODO: Write more?
  \caption{}
  \label{fig:sig2}
\end{figure}

\newpage
\section{Posterior distributions for sample 1 marker 1}
\begin{figure}[H]
  \begin{center}  % 6 x 5
    \imgtemplate{dden/dden_i1_j1.pdf}{\imscale}
  \end{center}
  % TODO: Write more?
  \caption{}
  \label{fig:ddi1j1}
\end{figure}

\newpage
\section{Posterior distributions for sample 1 marker 2}
\begin{figure}[H]
  \begin{center}  % 6 x 5
    \imgtemplate{dden/dden_i1_j2.pdf}{\imscale}
  \end{center}
  % TODO: Write more?
  \caption{}
  \label{fig:ddi1j2}
\end{figure}

\newpage
\section{Posterior distributions for sample 2 marker 1}
\begin{figure}[H]
  \begin{center}  % 6 x 5
    \imgtemplate{dden/dden_i2_j1.pdf}{\imscale}
  \end{center}
  % TODO: Write more?
  \caption{}
  \label{fig:ddi2j1}
\end{figure}

\newpage
\section{Posterior distributions for sample 2 marker 2}
\begin{figure}[H]
  \begin{center}  % 6 x 5
    \imgtemplate{dden/dden_i2_j2.pdf}{\imscale}
  \end{center}
  % TODO: Write more?
  \caption{}
  \label{fig:ddi2j2}
\end{figure}

\newpage
\section{Posterior distributions for sample 3 marker 2}
\begin{figure}[H]
  \begin{center}  % 6 x 5
    \imgtemplate{dden/dden_i3_j1.pdf}{\imscale}
  \end{center}
  % TODO: Write more?
  \caption{}
  \label{fig:ddi1j6}
\end{figure}

\newpage
\section{Posterior distributions for sample 3 marker 2}
\begin{figure}[H]
  \begin{center}  % 6 x 5
    \imgtemplate{dden/dden_i3_j2.pdf}{\imscale}
  \end{center}
  % TODO: Write more?
  \caption{}
  \label{fig:ddi2j6}
\end{figure}

\newpage
\section{Missing Mechanisms}
\begin{figure}[H]
  \begin{center}  % 6 x 5
    \begin{tabular}{ccc}
      \includegraphics[scale=0.35]{\imgdir/phi0/img/missmech_1.pdf} &
      \includegraphics[scale=0.35]{\imgdir/phi0/img/missmech_2.pdf} &
      \includegraphics[scale=0.35]{\imgdir/phi0/img/missmech_3.pdf} \\
    \end{tabular}
  \end{center}
  \caption{}
\label{fig:missmech}
\end{figure}

% Uncomment if using bibliography:
% \bibliography{sim}

\end{document}
